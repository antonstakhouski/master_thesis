\part*{Заключение}
\addcontentsline{toc}{part}{Заключение}

В ходе исследования был проведен обширный обзор аналогов системы. Был проведен
хронологический обзор основных этапов в развитии оффлайн и онлайн симуляции,
проведен тщательный анализ современного состояния проблемы. Результаты
исследования были систематизированы и представлены в виде сводной таблицы.
Каждый из рассмотренных классов алгоритмов обладает своими преимуществами и недостатками,
поэтому при разработке системы было решено использовать комбинацию методов для
объединения их преимуществ и минимизации недостатков каждого из методов.

Для более полного понимания объекта исследования был проведен анализ основных
физических процессов, управляющих поведением огня и визуальными характеристиками
пламени. Полученные наблюдения нашли отражение в выбранных методах симуляции и
рендеринга. Для моделирования стохастической природы огня были использована
система частиц, нечеткие границы и эффект полупрозрачности пламени были
достигнуты за счет использования тестурных сплэтов. Эффект движения пламени в
области с низким давлением, вызывающий колыхание языков пламени, лег в основу
алгоритма анимации огня.

При выборе инструментов для создания симулятора был проведен анализ множества
движков, фреймворков и библиотек. Для создания симулятора был выбран графический
интерфейс OpenGL как инструмент, предоставляющий оптимальный баланс между
возможностями для рендеринга сцены и уровню контроля над ресурсами системы.

Использованная комбинация техник и инструментов позволила создать
производительную систему, выполняющую эффектную анимацию и визуализацию огня.
Разработанная система предоставляет близкие к физическим методам визуальные
результаты, однако требует существенно меньших затрат аппаратных ресурсов.
Данная система может быть использована в видеоиграх для симуляции свободно
питаемого пламени, например, костров и факелов.

Данная система будет использована в качестве основы для будущих исследований. В
дальнейших работах планируется добавить моделирование распространения огня и
интегрировать пламя в систему освещения сцены.
