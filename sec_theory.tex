\chapter{Теоретические основы динамической симуляции огня}

В общем случае задача симуляции огня может быть разбита на три
непересекающихся подзадачи~\cite{Perry94synthesizingflames}:
\begin{itemize}
	\item моделирование;
	\item анимация;
	\item визуализация.
\end{itemize}

В первую очередь необходимо выбрать подходящую внутреннию структуру, или модель,
для симуляции. Далее, требуется выбрать способ анимации --- метод, с помощью
которого будет происходить взаимодействие с моделью. Техника анимации служит для
того, чтобы оживить модель, привести ее в движение. Наконец, модель и ее
анимацию необходимо отрисовать на экране, используя для этого некоторые
примитивы визуализации (полигоны, текстуры, сферы, воксели и т.п.). В
последующих разделах будут последовательно рассмотрены каждая из этих стадий, и
их взаимовлияние.

\section{Моделирование и визуализация огня}
Стохастические методы моделирования огня, такие как поля турбулентности и шумов
очень эффективны при создании реалистичного огня в 2D графике, но являются
крайне неэффективными по количеству операций для использования в 3D анимациях в
реальном времени, независимо от выбранной техники визуализации. Напротив,
эффективно релизованная объемная модель рендеринга, использующая воксели,
помогает достичь приемлемую для интерактивных приложений частоту кадров. Однако, при
использовании данной модели сложно добиться реалистичных результатов, поскольку
поверхность огня не имеет четких границ. Наиболее быстрым методом является
объемное моделирование с использованием полигонов для визуализации; отрисовка
полигонов происходит крайне быстро, однако полигоны не лучшее средство для
визуализации языков пламени, вихрящегося дыма и частиц сажи, и поэтому обычно полигоны
пораждают довольно грубые и низкокачественные результаты. Где-то между ними
находится моделирование с помощью систем частиц, данный метод может работать с
желаемой скоростью, в зависимости от выбранного масштаба, который выбирается из
расчета необходимого уровня детализации и выбранных техник анимации и
визуализации.
