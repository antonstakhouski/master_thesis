\documentclass[handout,professionalfonts]{beamer}

\usepackage{polyglossia}
\setmainlanguage[babelshorthands=true]{russian}
\setotherlanguage{english}
\defaultfontfeatures{Renderer=Basic, Ligatures=TeX}
\newfontfamily\cyrillicfonttt{CMU Typewriter Text}
\newfontfamily\cyrillicfont{CMU Sans Serif}
\setmainfont{CMU Sans Serif}
\setsansfont{CMU Sans Serif}
\setmonofont{CMU Typewriter Text}

% Обтекание рисунков
\usepackage{wrapfig}

% Зачем: Пакет для вставки картинок
% Примечание: Объяснение, зачем final - http://tex.stackexchange.com/questions/11004/why-does-the-image-not-appear
\usepackage{graphicx}
\DeclareGraphicsExtensions{.pdf,.png,.jpg,.jpeg}

% Зачем: Директория в которой будет происходить поиск картинок
\graphicspath{{figures/}}

% Зачем: Добавление подписей к рисункам. Рисунки нумеруются в пределах главы
% Почему: Пункт 4.4.8 Номер иллюстрации (таблицы) должен состоять из номера
% главы и порядкового номера иллюстрации (таблицы), разделенных точкой.
% Пункт 4.5 Слово «Рисунок», его номер и наименование иллюстрации печатают
% полужирным шрифтом, причем слово «Рисунок», его номер, а также пояснительные
% данные к нему – уменьшенным на 1–2 пункта размером шрифта.
\usepackage[nooneline,font=tiny]{caption}

% Зачем: Задание подписей, разделителя и нумерации частей рисунков
% Почему: Пункт 4.4.8 4.4.8 Иллюстрации и таблицы обозначают соответственно
% словами «Рисунок» и «Таблица» и нумеруют последовательно в пределах каждой главы.
\DeclareCaptionLabelFormat{stbfigure}{Рисунок #2}
\DeclareCaptionLabelFormat{stbtable}{Таблица #2}
\DeclareCaptionLabelSeparator{stb}{~--~}
\captionsetup{labelsep=stb}
\captionsetup[figure]{labelformat=stbfigure,justification=centering,labelfont=bf,textfont=bf,belowskip=-10pt}
\captionsetup[table]{labelformat=stbtable,justification=raggedright,format=hang,aboveskip=0pt}
% Зачем: группировка рисунков
\usepackage{subfig}
% Зачем: русские индексы для рисунков
\renewcommand\thesubfigure{\asbuk{subfigure}}

% Зачем: оформление библиографии через biblatex
\usepackage{csquotes}
\usepackage[
    sorting=none,
    backend=biber,
    defernumbers=true,
    bibstyle=gost-numeric-min,
    citestyle=gost-numeric,
    autolang=other,
    ] {biblatex}
% файл с библиографией
\addbibresource{bibliography_database.bib}

% Зачем: добавление отдельной категории для авторских публикаций
% Почему: Пункт 4.9 Сведения об использованных в магистерской диссертации
% источниках приводятся в разделе «Библиографический список», включающем
% подразделы «Список использованных источников» и «Список публикаций
% соискателя».
\DeclareBibliographyCategory{AuthorSources}
\addtocategory{AuthorSources}{mol_uch,55_sntk,56_sntk,56_sntk_vmip}

% Зачем: заменить разделитель между цитатами с ; на ,
% Почему: так принято. Пример есть в thesisby
\renewcommand{\multicitedelim}{\addcomma\space}

\setbeamertemplate{bibliography item}{\insertbiblabel}

% % Зачем: Добавление постфикса "--А" к авторским публикациям
% % https://tex.stackexchange.com/questions/540028/biblatex-add-postfix-to-label
% % Почему: Пункт 3.9 В списке использованных источников сведения об источниках
% % нумеруют арабскими цифрами, а в списке публикаций автора – арабскими цифрами,
% % которые через тире дополняются буквой «А.» («авторская») с точкой.
% %+++++++++++++++++++++++++++++++++++++++++++++++++++++++++++++++++++++++++++++++
\DeclareFieldFormat{labelprefix}{--#1}

\defbibenvironment{bibliography}
  {\list
     {\printtext[labelnumberwidth]{%
        \printfield{labelnumber}%
        \printfield{labelprefix}}}
     {\toggletrue{bbx:gostbibliography}% Более близкий к ГОСТ-2003 стиль
      \renewcommand*{\revsdnamepunct}{\addcomma}% Запятая перед инициалами
      \setlength{\labelwidth}{\labelnumberwidth}%
      \setlength{\leftmargin}{\labelwidth}%
      \setlength{\labelsep}{\biblabelsep}%
      \addtolength{\leftmargin}{\labelsep}%
      \setlength{\itemsep}{\bibitemsep}%
      \setlength{\parsep}{\bibparsep}}%
      \renewcommand*{\makelabel}[1]{\hss##1}}
  {\endlist}
  {\item}

\makeatletter
\renewbibmacro*{cite:comp:end}{%
  \usebibmacro{cite:dump}%
  \ifnumgreater{\value{cbx@tempcntb}}{-1}
    {\multicitedelim}
    {}%
  \printtext[bibhyperref]{%
    \printfield{labelnumber}%
    \printfield{labelprefix}}}

\renewbibmacro*{cite:comp:inset}{%
  \usebibmacro{cite:dump}%
  \ifnumgreater{\value{cbx@tempcntb}}{-1}
    {\multicitedelim}
    {}%
  \printtext[bibhyperref]{%
    \printfield{labelnumber}%
    \printfield{labelprefix}%
    \printfield{entrysetcount}}%
  \setcounter{cbx@tempcntb}{-1}}

\renewbibmacro*{cite:dump}{%
  \ifnumgreater{\value{cbx@tempcnta}}{0}
    {\ifnumgreater{\value{cbx@tempcnta}}{1}
       {\bibrangedash}
       {\multicitedelim}%
     \bibhyperref[\cbx@lastkey]{%
       \printtext[labelnumber]{\cbx@lastnumber}%
       \ifdef\cbx@lastprefix
         {\printtext[labelprefix]{\cbx@lastprefix}}
         {}}}
    {}%
  \setcounter{cbx@tempcnta}{0}%
  \global\undef\cbx@lastprefix}
\makeatother
%+++++++++++++++++++++++++++++++++++++++++++++++++++++++++++++++++++++++++++++++
\usepackage{lipsum}

% Тема презентации
\usetheme{AnnArbor}

% Титульный лист
\title{Динамическая симуляция объемного огня}
\author{Стаховский Антон}
\author[Стаховский~А.В.]{Стаховский~А.В. \\[1ex]
{\small \emph{Научный руководитель}: Кукин~Д.П., к.т.н., доцент}}
\institute[БГУИР]{Белорусский государственный университет информатики и радиоэлектроники}
\date{\today}


