\documentclass[professionalfonts]{beamer}

\usepackage{polyglossia}
\setmainlanguage[babelshorthands=true]{russian}
\setotherlanguage{english}
\defaultfontfeatures{Renderer=Basic, Ligatures=TeX}
\newfontfamily\cyrillicfonttt{CMU Typewriter Text}
\newfontfamily\cyrillicfont{CMU Sans Serif}
\setmainfont{CMU Sans Serif}
\setsansfont{CMU Sans Serif}
\setmonofont{CMU Typewriter Text}

% Зачем: Пакет для вставки картинок
% Примечание: Объяснение, зачем final - http://tex.stackexchange.com/questions/11004/why-does-the-image-not-appear
\usepackage{graphicx}
\DeclareGraphicsExtensions{.pdf,.png,.jpg,.jpeg}

% Зачем: Директория в которой будет происходить поиск картинок
\graphicspath{{figures/}}

% Зачем: Добавление подписей к рисункам. Рисунки нумеруются в пределах главы
% Почему: Пункт 4.4.8 Номер иллюстрации (таблицы) должен состоять из номера
% главы и порядкового номера иллюстрации (таблицы), разделенных точкой.
% Пункт 4.5 Слово «Рисунок», его номер и наименование иллюстрации печатают
% полужирным шрифтом, причем слово «Рисунок», его номер, а также пояснительные
% данные к нему – уменьшенным на 1–2 пункта размером шрифта.
\usepackage[nooneline,figurewithin=section,font=tiny]{caption}

% Зачем: Задание подписей, разделителя и нумерации частей рисунков
% Почему: Пункт 4.4.8 4.4.8 Иллюстрации и таблицы обозначают соответственно
% словами «Рисунок» и «Таблица» и нумеруют последовательно в пределах каждой главы.
\DeclareCaptionLabelFormat{stbfigure}{Рисунок #2}
\DeclareCaptionLabelFormat{stbtable}{Таблица #2}
\DeclareCaptionLabelSeparator{stb}{~--~}
\captionsetup{labelsep=stb}
\captionsetup[figure]{labelformat=stbfigure,justification=centering,labelfont=bf,textfont=bf,belowskip=-10pt}
\captionsetup[table]{labelformat=stbtable,justification=raggedright,format=hang,aboveskip=0pt}
% Зачем: группировка рисунков
\usepackage{subfig}
% Зачем: русские индексы для рисунков
\renewcommand\thesubfigure{\asbuk{subfigure}}

% Тема презентации
\usetheme{AnnArbor}

% Титульный лист
\title{Динамическая симуляция объемного огня}
\author{Стаховский Антон}
\author[Стаховский~А.В.]{Стаховский~А.В. \\[1ex]
{\small \emph{Научный руководитель}: Кукин~Д.П., к.т.н., доцент}}
\institute[БГУИР]{Белорусский государственный университет информатики и радиоэлектроники}
\date{\today}


