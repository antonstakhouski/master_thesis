\part*{Заключение}
\addcontentsline{toc}{part}{Заключение}

В ходе исследования предмета диссертации был проведен обзор большого количества
работ по данной теме. Анализ литературных источников показал наличие множества
кроссдисциплинарных связей в решениях данной проблемы. Поведение реального огня
описывается физикой, в том числе термодинамикой, создание симулятора тесно
связано с математическим моделированием и компьютерной графикой. Также
встречаются и менее очевидные решения, которые позволяют добиться интересных
результатов за счет использования методов и приемов из, на первый взгляд, не
связанных дисциплин. Например, использование алгоритма косяка птиц из теории роевого
поведения позволило создать согласованное движении частиц в языке
пламени. Таким образом симуляция огня является
сложной комплексной задачей, которая в данный момент не может быть решена
полностью. Для успешного решения задачи симуляции огня необходима разработка
специализированных решений, сфокусированных на моделировании ограниченного числа
атрибутов огня.

Данная работа направлена на создание трехмерной симуляции огня, которая может
быть использована в трехмерных видеоиграх. В ходе создания симулятора
приоритетными задачами являлись оптимизация использования вычислительных
ресурсов и улучшение визуальной составляющей симуляции.\break{}Разработанная система
показывает хорошую производительность, однако показывает не слишком реалистичные
визуальные результаты. Для устранения недостатков визуализации могут быть
использованы техники процедурного\break{}моделирования текстур, например, шум Перлина.

В дальнейших исследованиях по данной работе планируется уделить~\break{}внимание
алгоритмам распространения фронта огня, взаимодействию огня с окружающими
объектами и окружающей средой.
