\part*{Общая характеристика работы}
\addcontentsline{toc}{part}{Общая характеристика работы}

\textbf{Объект и предмет исследования}

\emph{Объектом исследования} является огонь в трехмерной графике как один из
элементов трехмерной сцены.

\emph{Предметом исследования} является динамическая симуляция объемного огня в
режиме реального времени.

\textbf{Цель и задачи исследования}

\emph{Цель исследования} --- разработка системы динамической симуляции
трехмерного огня.

\textbf{Задачи исследования:}

\begin{enumerate}
	\item Обзор и анализ научных работ по современным алгоритмам анимации огня и
        трехмерному рендерингу.
	\item Теоретические основы динамической симуляции огня.
	\item Обзор основных использованных алгоритмов.
	\item Реализация системы.
\end{enumerate}

\textbf{Связь с реальным сектором экономики}

На основе разработанной системы можно сформировать модули для популярных игровых
движков и графических редакторов, что обеспечит доступность системы для широкого
круга лиц и позволит распространять систему на соответствующих коммерческих
площадках. Полученные модули можно будет легко интергрировать в коммерческие
продукты.

\textbf{Апробация диссертации}

Результаты исследований по теме диссертации были представлены в виде доклада
''Современные алгоритмы моделирования аморфных объектов'' и представлены на
55-ой юбилейной научной конференции аспирантов, магистрантов и студентов БГУИР
в 2019 году.

\textbf{Публикация результатов исследований}

Результаты исследований были опубликованы в виде тезисов доклада на 55-ой
юбилейной научной конференции аспирантов, магистрантов и студентов
БГУИР~\cite{55_sntk}.  Также на основе полученных в ходе исследований
результатов была опубликована статья ''Анализ современных алгоритмов симуляции
огня''~\cite{mol_uch}.
