\part*{Введение}
\addcontentsline{toc}{part}{Введение}

Дым и огонь могут значительно влиять на визуальное восприятие объектов сцены, а
также влиять на свойства других объектов сцены. По этой причине огонь и дым
являются важными составляющими во многих прикладных областях, таких как
симуляция полетов, ландшафтный дизайн, анимация и киноиндустрия. Анимация и
визуализация данного явления сложной задачей и представляет определенный научный
интерес.

Симуляция трехмерного огня в режиме реального времени находит свое применение в
различных интерактивных приложениях.  Среди интерактивных приложений, анимации
огня наиболее востребованы в видеоиграх.В компьютерной графике
довольно часто требуется найти компромисс между скоростью и реализмом. Основной
проблемой рендеринга в реальном времени является поиск таких алгоритмов, которые
позволяют получить достаточную реалистичность, при которой частота кадров будет
не менее минимального порога.

В настоящее время разработчики игровых движков исследуют возможности
совместного использования новых подходов, таких как воксельная графика, с уже
устоявшимися на основе систем частиц. Задача оптимизации и улучшения
классических алгоритмов также остается актуальной.

Актуальные алгоритмы анимации огня и современные алгоритмы трехмерного
рендеринга являются предметом данной диссертации, цель которой -- создание
системы для динамической симуляции огня.

Компоненты данной системы в дальнейшем могут быть интегрированы в игровые
движки, трехмерные редакторы.
