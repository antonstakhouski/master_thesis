\chapter{Результаты экспериментов}

Компьютерная симуляция, описанная в главе~\ref{chap:implementation}, была
успешно разработана и реализована на основе кодовой базы из примеров,
предложенных в~\cite{LearnOGL}. Разработанный симулятор написан на языке C++ с
использованием программного интерфейса OpenGL 4.5. Программа была скомпилирована
с помощью компилятора GNU G++ версии 8.3.0. Окружение симуляции использует
операционную систему Debian 10 Buster, процессор Intel Core i5-5200U с частотой
2.7GHz, 8 ГБ оперативной памяти и видеокарту Intel(R) HD Graphics 5500.

В анимации изображения быстро сменяют друг друга на экране, при этом их их
содержание изменяется незначительно. Данный эффект используется для того, чтобы
заставить человеческий глаз воспринимать изменения как перемещение объектов. Чем
выше частота кадров, тем плавнее воспринимается движение объектов, в то время
как при низкой частоте кадров перемещения выглядят дергано. В фильмах и
телевидении используется частота в 24 кадра/секунду и 30 кадров/секунду
соответственно. Высокая частота кадров крайне важна в видеоиграх, где частота
кадров сильно влияет на игровой процесс. В консольных видеоиграх средняя частота
кадров составляет 30 кадров в секунду, рекомендуемой частотой ПК игр является 60
кадров в секунду.

Одним из важных экспериментов является поиск максимального количества частиц в
системе, при котором сохраняется приемлемая частота кадров. Для данного теста
было выполнено несколько опытов, в ходе которых максимальное количество частиц в
системе принимало значения в 5000, 12500, 25000, 50000 частиц. Результаты
представлены в таблице~\ref{table:amountBench}.
\begin{table}[htb]
\caption{Зависимость частоты кадров от количества частиц в системе}%
\label{table:amountBench}
\centering
\small
\begin{tabular}{| l | l |}
    \hline
    Количество частиц & Средняя частота кадров \\
    \hline
    5000 &  60,00 \\
    \hline
    10000 & 58,46 \\
    \hline
    15000 & 50,62 \\
    \hline
    25000 & 31,94 \\
    \hline
    50000 & 15,85 \\
    \hline
\end{tabular}
\end{table}

Как видно из таблицы~\ref{table:amountBench}, при 25000 частиц в системе
симуляция работает с еще приемлемой частотой кадров. Измерение нагрузки ЦП и ГП
при моделировании 25000 частиц показало всего лишь 26\% загрузку ГП и 100\%
загрузку ЦП. Таким образом, у системы имеется потенциал для увеличения
производительности, однако, для этого требуется выполнить оптимизацию
алгоритмов, работающих на ЦП.
