\part*{Введение}
\addcontentsline{toc}{part}{Введение}

Дым и огонь могут значительно влиять на визуальное восприятие объектов сцены, а
также влиять на свойства других объектов сцены. По этой причине огонь и дым
являются важными составляющими во многих прикладных областях, таких как
симуляция полетов, ландшафтный дизайн, анимация и киноиндустрия. В настоящее
время крайне популярен жанр фэнтези, в котором зачастую фигурируют множество
огненных и огнедышащих существ, таких как драконы, ифриты, птицы феникс. Для
создания данных существ в видеоиграх и кино, требуется моделировать поведение
огня, которому не существует аналога в реальном мире.  Из-за широкого спектра
художественных требований, иногда требуется, чтобы огонь вел себя как горящий
газ, в других ситуациях нужно, чтобы он вел себя как горящая жидкость либо имел
и вовсе фантастический вид. Часто требуется, чтобы огонь взаимодействовал с
окружающими его твердыми телами, а также водой и ветром.

Художники и дизайнеры могут составить крайне детализированный визуальный концепт
огня и описание его поведения, однако зачастую очень сложно преобразовать
данные идеи в соответствующие математические модели для симуляции огня в
графических сценах. Анимация и визуализация данного явления сложной задачей и
представляет определенный научный интерес.

Первая широко известная работа по симуляции огня была опубликована в 1982 году,
и полученная модель была использована в фильме ''Звездный путь 2: Гнев Хана''
для создания сцены взрыва генезис-бомбы. С развитием вычислительной техники и
активные научные исследования позволили значительно увеличить визуальное
качество эффектов в кино. Современное состояние работ в данной области можно
увидеть в фильме ''Хоббит: Битва пяти воинств'' в сцене, где Смауг сжигает
пламенем Эсгарот. При создании эффектов для кинофильмов главную роль играет
визуальное качество эффектов, а также гибкость и простота в настройке
параметров, для моделирования огня с необходимым визуальным стилем и поведением.
Скорость симуляции является вторичной, создание кадра может занимать несколько
часов. Современные алгоритмы, используемые в кино позволяют создавать
высокореалистичный огонь, однако применение данных алгоритмов для симуляции в
режиме реального времени затруднительно. Например, в видеоиграх необходимо
успевать рассчитывать и отрисовывать 60 и более кадров каждую секунду. Малое
время на отрисовку кадра и необходимость взаимодействия огня с объектами
окружающей среды приводят накладывают дополнительные ограничения на выбор
алгоритмов симуляции.

Симуляция трехмерного огня в режиме реального времени находит свое применение в
различных интерактивных приложениях.  Среди интерактивных приложений, анимации
огня наиболее востребованы в видеоиграх. В видеоиграх необходимость симуляции
огня была с самого момента их появления, однако, всего два десятилетия назад
стало возможным использовать огонь в трехмерных сценах.  В компьютерной графике
довольно часто требуется найти компромисс между скоростью и реализмом. В
приложениях реального времени, скорости отрисовки отдается наибольший приоритет;
увеличенный реализм бесполезен, если частота кадров не дотягивает до
определенного уровня. Поэтому основной проблемой рендеринга в реальном времени
является поиск таких алгоритмов, которые позволяют получить достаточную
реалистичность, при которой частота кадров будет не менее минимального порога.

Для каждого из желаемых атрибутов необходимо выбрать алгоритм, который оптимален
для наших целей. Такая оптимальность зачастую приводит к созданию различных
ухищрений для достижения желаемого эффекта. Основная идея в том, что если
результаты визуально приемлемые, и симуляция идет достаточно быстро, практически
никто не заметит особенной разницы. В этом нет никакого стыда, поскольку такие
ухищрения являются частью индустрии компьютерной графики с самого ее появления.

За последние 10 лет, значительно возросла производительность компьютеров,
особенно стремительно развивались графические карты. Современное аппаратное
обеспечение компьютеров, сделало возможным применение более сложных моделей,
использование алгоритмов анимации и рендеринга, использование которых раньше
было невозможно из-за аппаратных ограничений. В частности это активизировало
интерес рендерингу с помощью воксельной графики, некоторые из алгоритмов
воксельной графики были разработаны более 30 лет назад, однако, их использование
было невозможно либо крайне ограниченно из-за требований к оперативной памяти. В
актуальных публикациях используются уравнения Навье-Стокса для вычисления
скорости, плотности и температуры частиц огня.  Данный метод позволяет добиться
высокой реалистичности анимации, однако имеет высокую вычислительную сложность
по количеству операций.

В настоящее время разработчики игровых движков исследуют возможности
совместного использования новых подходов, таких как воксельная графика, с уже
устоявшимися на основе систем частиц. Задача оптимизации и улучшения
классических алгоритмов также остается актуальной.

Актуальные алгоритмы анимации огня и современные алгоритмы трехмерного
рендеринга являются предметом данной диссертации, цель которой -- создание
системы для динамической симуляции огня.

Компоненты данной системы в дальнейшем могут быть интегрированы в игровые
движки, трехмерные редакторы.
