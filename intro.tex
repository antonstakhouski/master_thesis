\part*{Введение}
\addcontentsline{toc}{part}{Введение}

Защита своих границ и граждан -- одна из наиболее приоритетных задач любого государства.
Страна, которая не уделяет достаточного внимания состоянию своих войск и вооружения не может гарантировать безопасность своих граждан и сохранение дальнейшее сохранение суверенитета.

За последние десятилетия военная техника и вооружение ушли далеко вперед.
Стали широко применяться различные датчики, спутниковые системы навигации, компьютерные сети, портативные компьютеры.
Благодаря внедрению автоматизации в расчеты, настройку оборудования, тестирование периферийных устройств эффективность вооруженных сил значительно возросла.
Во время эксплуатации военной техники внезапный отказ технических средств и локальной вычислительной сети может привести
к серьезным потерям личного состава, порче оборудования, потере преимущества на местности.
В таких условиях автоматизация процессов проведения тестирования является одной из наиболее приоритетных задач.

Исключительную важность во время проведения боевых действий\break представляет комплекс машин управления огнем, который служит для управления офицерским составом деятельностью своих подчиненных.

Целью данного дипломного проекта является разработка и реализация системы автоматизации процессов тестирования технических средств и каналов обмена данными в локальной сети.
Данная система в первую очередь ориентированна на использование артиллерийским дивизионом, но при небольших доработках программные модули могут быть также использованы в решениях для других армейских подразделений.

Для успешного выполнения поставленной цели, работа над проектом была разбита на следующие задачи:
\begin{itemize}
    \item выбор технологий, удовлетворяющих требованиям;
    \item разработка управляющего модуля;
    \item разработка алгоритмов функционального контроля навигационной системы;
    \item разработка алгоритмов функционального контроля метеостанции;
    \item разработка алгоритмов функционального контроля радиостанции;
    \item разработка алгоритмов функционального контроля прибора наблюдения разведчика <<Капонир>>;
    \item разработка алгоритмов функционального контроля лазерного \break целеуказателя-дальномера;
    \item разработка алгоритмов функционального контроля принтера;
    \item разработка алгоритмов журналирования результатов тестирования;
    \item разработка программы для просмотра журнала тестирования.
\end{itemize}

Система состоит из нескольких слабо связанных модулей.
