\chapter{Разработка системы динамической симуляции огня}

Фокус данного исследования направлен на создание реалистично выглядящего огня,
отрисовка которого происходит в режиме реального времени. Это включает в себя
создание компьютерной программы, которая служит для изучения и моделирования
внешнего вида и поведения огня, и при этом плавно работает в режиме реального
времени. В предложенной реализации моделируется поведение костра. Для
моделирования огня была выбрана система частиц, предложенная
в~\cite{reewes1983}.

Моделирование огня в режиме реального времени можно условно разделить на научное
и эстетическое. Несмотря на то, что обе модели стремятся создать реалистичное
пламя, цель и практическое приложение данных моделей отличается. Научные модели
нацелены на исследование динамики определенного феномена. Эти модели нацелены на
точное преставление огня с учетом лежащей в основе поведения огня физики,
включая термодинамику. Целью научных моделей является получение более глубоких
знаний о поведении огня. Научные модели часто используются в экспериментах,
изучении экологических проблем и проблем окружающей среды, военных симуляторах.

Напротив, эстетические модели более сосредоточены на
визуальном\break{}представлении огня, то есть на внешнем виде пламени на экране.
Целью эстетических моделей является воссоздание визуальных эффектов, присущих
огню, используя при этом сравнительно менее ресурсозатратные техники и методики
расчетов по сравнении с научными моделями. Таким образом, эстетические модели
лучше подходят для использования на компьютерах с ограниченными вычислительными
ресурсами а также для использования в приложениях реального времени, которые
крайне чувствительны к задержкам и падению производительности. Примером таких
приложений могут служить видеоигры.

Модель огня костра, разработанную в ходе данного исследования, можно отнести к
эстетической. Основное внимание при разработке системы уделялось воссозданию
поведения и внешнего вида огня, близких к их аналогам в реальном мире. При
создании симуляции использовались некоторые ухищрения, упрощения и ограничения,
которые имеют мало общего с физическими процессами, протекающими в реальном
огне.
