\chapter{Разработка системы динамической симуляции огня}%
\label{chap:implementation}

Фокус данного исследования направлен на создание реалистично выглядящего огня,
отрисовка которого происходит в режиме реального времени. Это включает в себя
создание компьютерной программы, которая служит для изучения и моделирования
внешнего вида и поведения огня, и при этом плавно работает в режиме реального
времени. В предложенной реализации моделируется поведение костра. Для
моделирования огня была выбрана система частиц, предложенная
в~\cite{reewes1983}.

Моделирование огня в режиме реального времени можно условно разделить на научное
и эстетическое. Несмотря на то, что обе модели стремятся создать реалистичное
пламя, цель и практическое приложение данных моделей отличается. Научные модели
нацелены на исследование динамики определенного феномена. Эти модели нацелены на
точное преставление огня с учетом лежащей в основе поведения огня физики,
включая термодинамику. Целью научных моделей является получение более глубоких
знаний о поведении огня. Научные модели часто используются в экспериментах,
изучении экологических проблем и проблем окружающей среды, военных симуляторах.

Напротив, эстетические модели более сосредоточены на
визуальном\break{}представлении огня, то есть на внешнем виде пламени на экране.
Целью эстетических моделей является воссоздание визуальных эффектов, присущих
огню, используя при этом сравнительно менее ресурсозатратные техники и методики
расчетов по сравнении с научными моделями. Таким образом, эстетические модели
лучше подходят для использования на компьютерах с ограниченными вычислительными
ресурсами а также для использования в приложениях реального времени, которые
крайне чувствительны к задержкам и падению производительности. Примером таких
приложений могут служить видеоигры.

Модель огня костра, разработанную в ходе данного исследования, можно отнести к
эстетической. Основное внимание при разработке системы уделялось воссозданию
поведения и внешнего вида огня, близких к их аналогам в реальном мире. При
создании симуляции использовались некоторые ухищрения, упрощения и ограничения,
которые имеют мало общего с физическими процессами, протекающими в реальном
огне.

\section{Моделирование огня}

При моделировании динамического поведения огня с помощью системы частиц
визуальное восприятие сцены сильно зависит от четырех важных атрибутов частиц:
формы, светимости, прозрачности и цвета. Общая форма огня зависит от размера
частиц и их анимации. Такие эффекты, как мерцание и разделение пламени, зависят
от динамического поведения частиц. Светимость частиц используется для симуляции
эффекта накаливания в пламени. Накаливание --- это эффект излучения нагретыми телами
света. Пламя представляет собой газовый феномен, который имеет высокую
температуру и выделяет большое количество энергии в окружающую среду. Таким
образом, сквозь пламя могут просвечиваться объекты, находящиеся по другую
сторону. Степень прозрачности частиц позволяет моделировать эффект
полупрозрачности, наблюдаемый в реальном пламени.

В данной работе для воссоздания цвета и внешнего вида огня были использованы
текстуры. Наблюдения показывают, что цвет пламени зависит от типа топлива и вида
окислителя, которые взаимодействуют в процессе горения. Углеродное топливо в
основном порождает языки пламени желтого либо оранжевого цветов, в то время как
различные газы порождают языки пламени голубоватого оттенка. Для упрощения
задачи в рамках исследования производилось моделирование только углеродного
топлива. Далее будет приведено описание разработанной модели на основе системы
частиц.

Разработанная система имеет три ключевых элемента: частицы, эмиттеры, менеджеры
эмиттеров. Данная структура основана на схеме, предложенной
в~\cite{Somasekaran2005UsingPS}. Частицы представлены как небольшие объекты без
четко определенного размера, структуры и траектории движения. Данная
недерминистичная природа частиц была достигнута путем использования
стохастических процессов для вычисления значения атрибутов эмиттера и частиц.
Данное решение позволило достаточно убедительно смоделировать хаотическую
природу огня. Общая схема алгоритма основана на схеме, предложенной Вудхаузом
в~\cite{Woodhouse} (рис.~\ref{fig:algorithm}).

\begin{figure}[htb]
	\centering
	\includegraphics[width=0.7\textwidth]{algorithm}
    \caption{Схема обновления частиц в кадре, предложенная Вудхаузом
    в~\cite{Woodhouse}}%
    \label{fig:algorithm}
\end{figure}

Частицы создаются эмиттерами и затем вводятся в систему частиц.\break{}Эмиттеры
являются случайными точками, расположенными в рамках указанных границ. По факту,
можно считать эмиттеры статическими частицами.\break{}Эмиттеры обладают
различными характеристиками, которые вносят вклад в динамику частиц и общий
внешний вид модели огня. Ключевыми атрибутами эмиттеров являются их позиция,
скорость эмиссии частиц, начальный вектор эмиссии, динамический список частиц и
энергия эмиттера.

Местоположение эмиттеров определяет, где в пределах системы частиц будут
создаваться эмиттеры. В предложенной реализации эмиттеры создаются внутри круга,
который обозначает радиус огня. Эмиттеры также управляют скоростью эмиссии
частиц и преобладающим направлением эмиссии. Скорость эмиссии или скорость
горения определяет число частиц, которые испускает каждый эмиттер в момент
времени. Как видно на рисунке~\ref{fig:partLayers}
\begin{figure}[htb]
	\centering
	\includegraphics[width=0.4\textwidth]{partLayers}
    \caption{На картинке представлен эффект появления слоев, состоящих из частиц
    испущенных в один момент времени}%
    \label{fig:partLayers}
\end{figure}
частицы могут появляться в виде слоев на экране. Это происходит потому, что все
частицы с постоянной скоростью генерируются у основания эмиттера, вектора
движения частиц также совпадают. Для устранения данного недостатка можно при
генерации частиц добавить им случайное смещение вдоль оси $y$, также можно
добавить небольшие случайные смещения к начальным векторам скорости частиц.
Доработанная система демонстрирует более реалистичные результаты и представлена
на рисунке~\ref{fig:partLayersFix}.
\begin{figure}[htb]
	\centering
	\includegraphics[width=0.4\textwidth]{partLayersFix}
    \caption{Уменьшение кластеризации частиц с помощью добавления случайных
    смещений}%
    \label{fig:partLayersFix}
\end{figure}
Эффект кластеризации может приводить к видимым разрывам в анимации из-за
существенного перепада количества частиц. Оптимальные значения для коэффициентов
смещения следует подбирать опытным путем.

Когда эмиттер создается ему присваивается случайное количество частиц, которые
хранятся в динамическом списке. Список изменяется при добавлении или удалении
частиц из системы. Как можно понять, производительность данного решения крайне
страдает из-за постоянных операций по управлению памятью. Решение данной
проблемы взято из~\cite{LearnOGL}. В данной работе предложено выделять память
заранее для каждого эмиттера, выделенная область ограничивает максимальное
количество частиц присутствующих в системе, которые были выделены одним
эмиттером. Для обновления атрибутов и дальнейшего рендеринга выбираются только
те частицы, которые имеют положительное значение оставшегося времени жизни. При
добавлении новой частицы в систему происходит поиск в списке уже неактивной
частицы и замена ее на новую.

Недостатками данной реализации являются:
\begin{enumerate}
    \item Необходимость определения максимального количества частиц в системе
    заранее. Это несет существенные затраты по памяти. Также при отсутствии
    свободных частиц происходит замена случайной частицы на новую, что может
    быть нежелательно.
    \item Высокая алгоритмическая сложность алгоритма поиска новой частицы.
    Алгоритм имеет линейную сложность ($O(n)$) в худшем случае. В алгоритме есть
    небольшая оптимизация. Алгоритм хранит индекс последней неиспользованной
    частицы, и пытается на следующем шаге начинать поиск с нее. Эта оптимизация
    слабо помогает при постоянном наличии большого числа частиц в системе.
    Если у частиц еще значительно варьируется время жизни --- практически всегда
    выполняется линейный поиск.
\end{enumerate}

Наибольшую проблему вызывал алгоритм обновления частиц. Он являлся бутылочным
горлышком системы. Для решения данной проблемы было разработано тривиальное, но
эффективное решение: в цикле обновления частиц индексы ''умерших'' частиц
помещаются в отдельный список. Данный список используется на следующем шаге
симуляции на этапе генерации новых частиц. Данное решение немного увеличило
затраты по памяти, однако позволило выполнять операцию поиска ''умерших'' частиц
за константное время ($O(1)$).

Вернемся к описанию эмиттеров. Энергия эмиттера определяет время жизни системы
частиц. Каждую итерацию энергия эмиттера уменьшается, пока не достигнет
минимального значения, обозначающего, что эмиттер иссяк. Иссякшие эмиттеры
убираются из системы вместе со всеми назначенными им частицами. Менеджер
эмиттеров содержит список всех эмиттеров в системе. Когда эмиттер создается, он
добавляется в динамический список. Менеджер эмиттеров каждую итерацию обновляет
все эмиттеры и удаляет иссякшие эмиттеры из системы. В
таблице~\ref{table:emitterAttribs} приведено описание всех атрибутов эмиттеров,
описанных на данный момент.
\begin{table}[htb]
\caption{Атрибуты эмиттера}%
\label{table:emitterAttribs}
\centering
\small
\begin{tabular}{| l | l |}
    \hline
    Атрибут & Описание \\
    \hline
    Позиция & Местоположение эмиттера в трехмерном пространстве \\
    Энергия & Время жизни эмиттера \\
    Скорость & Модуль начальной скорости испускаемых эмиттером частиц \\
    Направление & Единичный вектор, указывающий направление эмиссии частиц \\
    Радиус & Радиус эмиттера. Ограничивает область генерации частиц \\
    Список частиц & Динамически обновляемый список принадлежащих эмиттеру
    частиц \\
    \hline
\end{tabular}
\end{table}

В качестве структуры симулятора была выбрана схема, предложенная
в~\cite{Somasekaran2005UsingPS}. Общая структура симулятора представлена на
рисунке~\ref{fig:simStructure}.
\begin{figure}[htb]
	\centering
	\includegraphics[width=\textwidth]{simStructure}
    \caption{Иерархия объектов, использованная в разработанном симуляторе}%
    \label{fig:simStructure}
\end{figure}

При описании структуры симулятора, достаточного внимание не было уделено
описанию частиц. Частицы являются ключевыми объектами в симуляции. Они имеют
такие атрибуты как позицию, энергию (время жизни), цвет, скорость, ускорение и
размер.  Каждая частица появляется в системе недалеко от центра эмиттера, в
дальнейшем ее местоположение изменяется в течение всего времени жизни частицы.
Скорость частицы определяет ее следующее местоположение в системе. Для
улучшения визуального восприятия сцены, к частицам также применяется ускорение.
Как и эмиттеры, частицы также обладают атрибутом энергии, который определяет
время жизни частицы в системе. Когда частица исчерпала свою энергию, ее место
занимает новая, что позволяет оптимизировать операции по управлению памятью.
Цвет и размер частицы определяется на основе количества энергии частицы и играет
существенную роль в формировании визуального представления сцены. По мере
уменьшения энергии частицы ее размер, интенсивность цвета и прозрачность
уменьшаются. Это помогает создать эффект незаметного исчезновения частиц по
истечении их времени жизни. Краткий обзор всех атрибутов частиц представлен в
таблице~\ref{table:partAttribs}.
\begin{table}[htb]
\caption{Атрибуты частицы}%
\label{table:partAttribs}
\centering
\small
\begin{tabular}{| p{0.12\textwidth} | p{0.78\textwidth} |}
    \hline
    Атрибут & Описание \\
    \hline
    Позиция & Местоположение частицы в трехмерном пространстве \\
    Скорость & Вектор скорости частицы \\
    Ускорение & Вектор ускорения частицы \\
    Энергия & Представляет оставшееся время жизни частицы \\
    Цвет & Цвет частицы. Интенсивность и прозрачность уменьшаются по мере
    старения частицы \\
    Размер & Размер частицы. Размер частицы уменьшается по мере старения частицы \\
    \hline
\end{tabular}
\end{table}

Значения атрибутов частиц рассчитываются по формулам
(\ref{eq:life}--\ref{eq:color}).
\begin{align}
    \label{eq:life}
    \text{Life}_{t + \Delta{t}} &= \text{Life}_{t} - \Delta{t} \\
    \label{eq:scale}
    \text{Scale}_{t + \Delta{t}} &= \text{Scale}_{{t}_{0}} \cdot
    \frac{\text{Life}_{t + \Delta{t}}}{\text{Life}_{{t}_{0}}} \\
    \label{eq:color}
    \text{Color}_{t + \Delta{t}} &= \frac{\text{Life}_{t + 1}}{\text{Life}_{{t}_{0}}}
\end{align}
\begin{explanationx}
    \item [где] $\text{Life}_{t + \Delta{t}}$ --- значение энергии (оставшееся
        время жизни) частицы в следующем кадре;
    \item $\text{Life}_{t}$ --- текущее значение энергии частицы;
    \item $\Delta{t}$ --- величина интервала между кадрами;
    \item $\text{Scale}_{t + \Delta{t}}$ --- размер (коэффициент
        масштабирования) частицы  в следующем кадре;
    \item $\text{Scale}_{{t}_{0}}$ --- начальное значение размера частицы;
    \item $\text{Life}_{{t}_{0}}$ --- начальное значение энергии частицы;
    \item $\text{Color}_{t + \Delta{t}}$ --- значение цвета (интенсивность и
        прозрачность) в следующем кадре.
\end{explanationx}

Первый прототип системы представлен на рисунке~\ref{fig:firstProto}.
\begin{figure}[htb]
	\centering
    \includegraphics[width=\textwidth]{firstProto}
    \caption{Первый прототип системы}%
    \label{fig:firstProto}
\end{figure}
Данный прототип страдал от множества недостатков:
\begin{enumerate}
    \item Высокая загрузка ГП. Несмотря на сравнительно небольшое количество
        объектов загрузка ГП достигала 85\%. При этом в системе находилось всего
        лишь 500 частиц. Каждый кадр генерировалось по 20 частиц.
    \item Проблемы с текстурами. Ошибка с наложением текстур была исправлена в
        следующих версиях.
    \item Отсутствие должной анимации частиц. Все частицы летят строго в
        направлении эмиссии. Созданию анимации пламени будет посвящен отдельный
        раздел.
\end{enumerate}

Ключевой проблемой на данной стадии являлась чрезмерная загрузка ГП. Для
устранения этой проблемы была задействована техника инстансинга. Идеей данного
метода является уменьшение количества вызовов к ГП. Более подробно
эта техника описана в~\cite{OGLSuperbible,LearnOGL}. Данная техника хорошо
подходит для рендеринга множества однородных объектов, какими и являются
частицы. Вместо того, чтобы в цикле выставлять атрибуты частицы и делать вызов
для функции отрисовки частицы, атрибуты всех частиц записываются в массивы.
Рендеринг всех частиц осуществляется в помощью одного обращения к ГП. Реализация
данной техники позволило увеличить количество частиц в системе до 5000,
количество частиц создаваемых за кадр --- до 750. При этом нагрузка на ГП
снизилась до 28\%. Модифицированная система представлена на
рисунке~\ref{fig:proto2}.
\begin{figure}[htb]
	\centering
    \includegraphics[height=10cm]{proto2}
    \caption{Использование инстансинга для отрисовки частиц}%
    \label{fig:proto2}
\end{figure}

Увеличение максимального числа частиц и уменьшение их размера положительно
скажется на реалистичности анимации. Однако, оптимизация работы ГП
продемонстрировала недостатки алгоритмов, работающих на ЦП. Загрузка ЦП
достигала 100\% (симуляция выполняется в один поток).

Для решения данной проблемы был использована оптимизация алгоритма обновления
частиц, описанная выше в данном разделе. Данная оптимизация позволила снизить
среднюю загрузку ЦП до 18\%, что позволило выполнять симуляцию с частотой 60
кадров / секунду. Изменения можно увидеть на рисунке~\ref{fig:proto3}.
\begin{figure}[htb]
	\centering
    \includegraphics[width=0.4\textwidth]{proto3}
    \caption{Система, основанная на новом алгоритме поиска ''мертвых'' частиц}%
    \label{fig:proto3}
\end{figure}

Изменение высоты пламени и скорости движения частиц вызваны изменением алгоритма
поиска ''мертвых'' частиц. В прошлой версии алгоритма при отсутствии свободных
частиц добавления новой частицы не происходило, в новой версии происходит замена
случайной частицы при неудаче. В продемонстрированном примере часто происходит
удаление случайных частиц и замена их на новые. Из-за данного эффекта
увеличилась плотность частиц у основания пламени, визуально увеличилась скорость
частиц. Удаление случайных частиц позволило устранить эффект непрерывного столба
пламени, наблюдаемый на рисунке~\ref{fig:proto2}. Данный эффект положительно
сказался на визуальном восприятии сцены и будет использован в дальнейшем.

Дальнейшие улучшения были связаны с увеличением стохастичности симуляции. К
значениям многих атрибутов были добавлены случайные смещения, что позволило
уменьшить количество статических элементов симуляции. Система с внесенными
оптимизациями и улучшениями представлена на рисунке~\ref{fig:proto4}.
\begin{figure}[htb]
	\centering
    \includegraphics[width=0.4\textwidth]{proto4}
    \caption{Использование стохастических эффектов в симуляции}%
    \label{fig:proto4}
\end{figure}
Однако, наиболее заметным недостатком данной системы является отсутствие
анимации. Решение, использованное для анимации системы будет описано в следующем
разделе.

\section{Анимация пламени}

В ходе исследования различных алгоритмов для анимации огня, был проведен обзор
нескольких различных подходов, описание которых можно найти в
главе~\ref{chap:lit_review}. В общем случае, в момент времени $t$ новое
положение частицы зависит от ее скорости $\vec{v}(t)$, ускорения $\vec{a}$, и интервала
времени между кадрами $\Delta t$. Как было замечено ранее, начальный вектор
скорость частицы определяется вектором эмиссии конкретного эмиттера. Для расчета
следующего положения частицы $\vec{p}(t + \Delta{t})$, используется следующие
уравнения (\ref{eq:position},~\ref{eq:velocity}).
\begin{align}
  \label{eq:position}
  \vec{p}(t + \Delta{t}) &= \vec{p}(t) + \vec{v}(t) \cdot \Delta{t} \\
  \label{eq:velocity}
  \vec{v}(t + \Delta{t}) &= \vec{v}(t) + \vec{a}
\end{align}
\begin{explanationx}
    \item [где] $\vec{p}(t + \Delta{t})$ --- координаты частицы в следующий
        момент времени;
    \item $\vec{p}(t)$ --- текущие координаты частицы;
    \item $\vec{v}(t)$ --- скорость частицы в текущий момент времени;
    \item $\Delta{t}$ --- величина временного интервала между кадрами;
    \item $\vec{v}(t + \Delta{t})$ --- скорость частицы в следующий момент времени;
    \item $\vec{a}$ --- текущее ускорение частицы.
\end{explanationx}

Как видно из уравнений выше, движение частиц в системе является равноускоренным.
Частицы немного ускоряются по мере их движения. Для данной симуляции была
эмпирически подобрана величина ускорения, вычисляемая по следующей формуле
(\ref{eq:accelearation}).
\begin{equation}
  \label{eq:accelearation}
  \vec{a} = 0,02 \cdot \vec{v}_{0}
\end{equation}
\begin{explanationx}
    \item [где] $\vec{a}$ --- ускорение частицы;
    \item $\vec{v}_{0}$ --- начальная скорость частицы.
\end{explanationx}

Как можно заметить, описанные выше уравнения описывают движение частиц как
равноускоренное прямолинейное движение. Однако, как было описано в
разделе~\ref{section:firePerception}, воздействующие на огонь силы создают более
сложное движение частиц пламени. Движение частиц пламени может быть описано с
помощью уравнений Навье-Стокса для жидкостей~\cite{ForGames}.

Математически состояние жидкости в данный момент времени моделируется как поле
векторов скоростей: функции, присваивающей вектор скорости каждой точке в
пространстве. Это можно представить на примере воздуха, находящегося в комнате,
его скорость в различных точках будет отличаться из-за наличия источников тепла,
сквозного ветра и различных других факторов. Например, вектор скорости возле
радиатора будет направлен вверх из-за подъема теплых воздушных масс.
Распределение скоростей в комнате имеет сложную структуру, как можно заметить,
наблюдая за движением дыма от сигареты, либо за движением частиц пыли в воздухе.
Уравнения являются точным описанием эволюции поля скорости в течение времени.
Следующие уравнения (\ref{eq:ns_velocity},~\ref{eq:ns_density}) описывают
изменения скорости и плотности частиц в компактной нотации.
\begin{align}
    \label{eq:ns_velocity}
    \frac{\partial{\vec{v}}}{\partial{t}} &= -(\vec{v} \cdot \nabla)\vec{v} + \nu \nabla ^ 2 \vec{v} + \vec{f} \\
    \label{eq:ns_density}
    \frac{\partial{\rho}}{\partial{t}} &= -(\vec{v} \cdot \nabla)\rho + \kappa \nabla ^ 2 \rho + P
\end{align}
\begin{explanationx}
    \item [где] $\vec{v} = (v ^ 1, \ldots, v ^ n)$ --- векторное поле скорости;
    \item $t$ --- время;
    \item $\nabla$ --- оператор набла;
    \item $\nu$ --- коэффициент кинематической вязкости;
    \item $\vec{f}$ --- векторное поле массовых сил;
    \item $\rho$ --- плотность;
    \item $\kappa$ --- коэффициент динамической вязкости (сдвиговая вязкость);
    \item $P$ --- давление.
\end{explanationx}

Важным отличием реального огня, от описанной
выше модели, является постоянное движение и изменение формы языков пламени. В
статье~\cite{Harris} Том Харрис дает описание этому феномену. Данный эффект
происходит из-за того, что когда газы нагреваются, они становятся менее
плотными, чем окружающий их воздух. Поэтому газы двигаются в зону, где давление
ниже. Чтобы просимулировать данный эффект в объеме, покрывающем зону горения,
случайным образом выбираются точки. Эти случайные точки симулируют зоны с
низким давлением. Каждую итерацию эти точки заменяются на другие, также
выбранные случайным образом.

Когда частица попадает в систему, она движется в сторону ближайшей точки с
низким давлением. Таким образом, точки с низким давлением формируют острые языки
пламени, которые можно наблюдать в реальном мире. Данный алгоритм представлен на
рисунке~\ref{fig:animationAlgo}. Реализация данного алгоритма представлена на
рисунке~\ref{fig:protoFinal}.
\begin{figure}[htb]
	\centering
    \includegraphics[width=\textwidth]{animationAlgo}
    \caption{Алгоритм анимации частиц}%
    \label{fig:animationAlgo}
\end{figure}
\begin{figure}[htb]
	\centering
    \includegraphics[width=0.4\textwidth]{protoFinal}
    \caption{Реализация анимации частиц}%
    \label{fig:protoFinal}
\end{figure}

Одной из сложностей при реализации данного алгоритма была реализация быстрого
алгоритма поиска ближайшей к частицы точки с низким давлением. В версии на
рисунке~\ref{fig:protoFinal} используется 5000 частиц ($m$) и 500 точек низкого
давления ($n$). Линейный поиск по такой системе имеет сложность
$O(m) \cdot O(n)$.
В оптимизированном методе используется алгоритм быстрой сортировки
($O(n \log n)$) для упорядочивания точек низкого давления по величине
координаты $y$, в отсортированном массиве выполняется бинарный поиск
($O(\log_{2} n)$) ближайшей к частице точки низкого давления (сравнение по
расстоянию до точки и значению координаты $y$). Поиск ближайшей точки низкого
давления по новому алгоритму имеет сложность
$O(n \log n) + O(m) \cdot O(\log_{2} n)$.

\section{Рендеринг пламени}

Результаты, представленные на рисунке~\ref{fig:protoFinal} имеют слабо отражают
визуальные характеристики огня, такие как нечеткость границ, прозрачность
пламени. Пламя, изображенное на рисунке выглядит довольно нереалистично и
дискретно. Для устранения данных недостатков было решено воспользоваться
техникой текстурного сплэттинга, описанной в
разделе~\ref{sec:modellingAndRendering}. Для реализации данного метода был
выбран текстурный сплэт, представленный на рисунке~\ref{fig:fireTexture}.

\begin{figure}[htb]
	\centering
    \includegraphics[width=0.3\textwidth]{fireTexture}
    \caption{Текстурный сплэт, используемый в симуляторе}%
    \label{fig:fireTexture}
\end{figure}

Для реализации сцены, представленной на рисунке~\ref{fig:protoSplats},
используемые для рендеринга частиц примитивы были заменены с кубических на
квадратные. Это позволило значительно уменьшить дискретность изображения,
однако, пламя теряет реалистичность при перемещении камеры вдоль оси $z$.

\begin{figure}[htb]
	\centering
    \includegraphics[width=0.15\textwidth]{simSplats}
    \caption{Использование текстурных сплэтов для рендеринга частиц}%
    \label{fig:protoSplats}
\end{figure}

Текстуры предоставляют множество дополнительной визуальной\break{}информации, что
позволило увеличить размер частиц и уменьшить количество частиц в системе, при
этом визуальное качество сцены не пострадало.

\section{Результаты экспериментов}

Компьютерная симуляция, описанная ранее в этой главе, была
успешно разработана и реализована на основе кодовой базы из примеров,
предложенных в~\cite{LearnOGL}. Разработанный симулятор написан на языке C++ с
использованием программного интерфейса OpenGL 4.5. Программа была скомпилирована
с помощью компилятора GNU G++ версии 8.3.0. Окружение симуляции использует
операционную систему Debian 10 Buster, процессор Intel Core i5--5200U с частотой
2.7ГГц, 8 ГБ оперативной памяти и видеокарту Intel HD Graphics 5500.

В анимации изображения быстро сменяют друг друга на экране, при этом их их
содержание изменяется незначительно. Данный эффект используется для того, чтобы
заставить человеческий глаз воспринимать изменения как перемещение объектов. Чем
выше частота кадров, тем плавнее воспринимается движение объектов, в то время
как при низкой частоте кадров перемещения выглядят дергано. В фильмах и
телевидении используется частота в 24 кадра/секунду и 30 кадров/секунду
соответственно. Высокая частота кадров крайне важна в видеоиграх, где частота
кадров сильно влияет на игровой процесс. В консольных видеоиграх средняя частота
кадров составляет 30 кадров в секунду, рекомендуемой частотой ПК игр является 60
кадров в секунду.

Одним из важных экспериментов является поиск максимального количества частиц в
системе, при котором сохраняется приемлемая частота кадров. Для данного теста
было выполнено несколько опытов, в ходе которых максимальное количество частиц в
системе принимало значения в 5000, 12500, 25000, 50000 частиц. Результаты
представлены в таблице~\ref{table:amountBench}.
\begin{table}[htb]
\caption{Зависимость частоты кадров от количества частиц в системе}%
\label{table:amountBench}
\centering
\small
\begin{tabular}{| l | l |}
    \hline
    Количество частиц & Средняя частота кадров \\
    \hline
    5000 &  60,00 \\
    \hline
    10000 & 58,46 \\
    \hline
    15000 & 50,62 \\
    \hline
    25000 & 31,94 \\
    \hline
    50000 & 15,85 \\
    \hline
\end{tabular}
\end{table}

Как видно из таблицы~\ref{table:amountBench}, при 25000 частиц в системе
симуляция работает с еще приемлемой частотой кадров. Измерение нагрузки ЦП и ГП
при моделировании 25000 частиц показало всего лишь 26\% загрузку ГП и 100\%
загрузку ЦП. Таким образом, у системы имеется потенциал для увеличения
производительности, однако, для этого требуется выполнить оптимизацию
алгоритмов, работающих на ЦП.

Для более объективной оценки производительности системы стоит сравнить ее с
наиболее близкими аналогами. Среди аналогичных работ, проведенных за последние 5
лет, наиболее близким аналогом является система, предложенная Юзефом Хладки и
Романом Журиковичем в 2018 году в статье~\cite{turbulence}. Краткое описание
работы можно найти в разделе~\ref{sec:online_sim}. Пример работы системы
представлен на рисунке~\ref{fig:turbulence}.
\begin{figure}[htb]
	\centering
    \includegraphics[width=\textwidth]{turbulence}
    \caption{Результаты, полученные в работе~\cite{turbulence}}%
    \label{fig:turbulence}
\end{figure}

Авторы выполняли симуляцию на сопоставимом по производительности ПК. Симуляция
выполнялась на ноутбуке с процессором Intel Core i3 350m с частотой 2.26ГГц (2
ядра), 4ГБ ОЗУ и видеокартой ATI Radeon 5145 с 512МБ ОЗУ.

В проведенном автором эксперименте использовался один эмиттер, содержащий один
сплайн. Сплайн содержал до 15 сегментов, содержащих по 100 частиц каждый, таким
образом в системе находилось одновременно не более 1500 частиц. Такая
конфигурация позволяла выполнять симуляцию со скоростью более 30 кадров в
секунду. Увеличение количества сегментов либо количества частиц приводило к
резкому падению частоты кадров.

Таким образом, разработанная система значительно превосходит по
производительности систему, разработанную Хладки и Журиковичем, позволяя
получать более детализированные результаты (рис.~\ref{fig:protoSplats}) с более
высокой частотой кадров.

К преимуществам работы, предложенной в статье~\cite{turbulence}, можно отнести
более точную симуляцию физических процессов, протекающих в огне, и возможность
более гибкой настройки пользователем параметров симуляции.

\textbf{Выводы по главе 3:}
\addcontentsline{toc}{section}{Выводы по главе 3}
\begin{enumerate}
    \item Реализованная система имеет множество оптимизаций для улучшения
        производительности, что позволяет выполнять симуляцию десятков тысяч
        частиц в системе в реальном времени даже на относительно слабом
        аппаратном обеспечении.
    \item Производительность разработанной системы существенно превосходит результаты,
        полученные с помощью более сложных математических моделей, при этом
        система демонстрирует сопоставимые визуальные результаты.
    \item Разработанная система представляет собой хорошо отлаженный и
        оптимизированный симулятор, который может быть использован для
        дальнейших исследований в области симуляции огня.
\end{enumerate}
