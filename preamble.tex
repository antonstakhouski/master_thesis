% Зачем: Определяет класс документа (то, как будет выглядеть документ)
% Примечание: параметр draft помечает строки, вышедшие за границы страницы, прямоугольником, в финальной версии его нужно удалить.
% Почему: Пункт 4.1 Материалы автореферата и пояснительной записки магистерской диссертации печатаются с помощью
% компьютера на одной стороне листа белой бумаги формата А4 (210 х 297 мм).
\documentclass[a4paper,14pt,russian,oneside,draft]{extreport}
% \documentclass[a4paper,14pt,russian,oneside,final]{extreport}

% Для мультиязыковой поддержки
\usepackage{polyglossia}

% Установка языков
\setdefaultlanguage{russian}
\setmainlanguage{russian}
\setotherlanguage{english}

% Зачем: так принято в русском языке
\PolyglossiaSetup{russian}{indentfirst=true}

% Зачем: Установка основных шрифтов
% Примечание: установка лигатур нужны для правильного отображения тире, кавычек и прочего
% Почему: Пункт 4.2 ...При этом рекомендуется использовать шрифты типа Times New Roman размером 14 пунктов.
\setmainfont[Ligatures=TeX]{Times New Roman}
\setmonofont[Ligatures=TeX]{Courier New}

% Необходимо для кириллических шрифтов
\newfontfamily{\cyrillicfont}[Ligatures=TeX]{Times New Roman}
\newfontfamily{\cyrillicfonttt}[Ligatures=TeX]{Courier New}

% Зачем: Межстрочный интервал в 18pt точно
% Почему: Пункт 4.2. Межстрочный интервал должен составлять 18 пунктов.
\usepackage{leading}
\leading{18pt}

% Зачем: Настраивает отступы от границ страницы.
% Почему: Пункт 4.2. ...соблюдая следующие размеры полей: левое – 30 мм, правое – 10 мм, верхнее и нижнее – 20 мм.
\usepackage[left=3cm,top=2cm,right=1cm,bottom=2cm]{geometry}

% Зачем: Длинна, примерно соответствующая 5 символам
% Почему: Требования содержат странное требование про отсупы в 5 символов (для немоноширинного шрифта :| )
% TODO: без понятия, зачем это
\newlength{\fivecharsapprox}
\setlength{\fivecharsapprox}{6ex}

% Зачем: Добавляет отступы для абзацев.
% Почему: Пункт 2.1.3 Требований по оформлению пояснительной записки.
% TODO: без понятия, зачем это
\usepackage{indentfirst}
\setlength{\parindent}{\fivecharsapprox} % Примерно соответствует 5 символам.

% Зачем: Установить для large размер в 16пт
% Почему: Пункт 4.3.1 Положения о магистерской диссертации
\makeatletter
\renewcommand\large{\@setfontsize\large{16pt}{18}} % 18 -- baseline skip
\makeatother

% Зачем: для оформления введения и заключения, они должны быть выровнены по центру.
% Почему: Пункт 4.3.1 Заголовки структурных частей пояснительной записки магистерской диссертации печатают прописными
% буквами в середине строк, используя полужирный шрифт с размером на 1–2 пункта больше, чем шрифт в основном тексте. Так
% же печатают заголовки глав.
\makeatletter
\renewcommand\part{%
  \clearpage\@startsection{part}{-1}%
    {\fivecharsapprox}%
    {-1em plus-1ex minus-.2ex}%
    {1em plus.2ex}%
    {\centering\hyphenpenalty=10000\normalfont\large\bfseries\MakeUppercase}%
    }
\makeatother

% Зачем: Задает стиль заголовков глав жирным шрифтом, прописными буквами, без точки в конце
% Почему: Пункт 4.3.1 Заголовки структурных частей пояснительной записки магистерской диссертации печатают прописными
% буквами в середине строк, используя полужирный шрифт с размером на 1–2 пункта больше, чем шрифт в основном тексте. Так
% же печатают заголовки глав.
% Пункт 4.4.2 Номер главы ставят после слова «ГЛАВА», после номера точку не ставят. Заголовок главы печатают с новой
% строки, следующей за номером главы.
\usepackage{titlesec}
\titleformat{\chapter}[display]
{\normalfont\large\centering\bfseries}
{\MakeTextUppercase\chaptertitlename\ \thechapter}
{.5em}
{\MakeTextUppercase}

\titlespacing*{\chapter}
  {0pt}{0pt}{1em}

% Зачем: Задает стиль заголовков разделов
% Почему: Пункт 4.3.1 Заголовки разделов печатают строчными буквами (кроме первой прописной) с абзацного отступа
% полужирным шрифтом с размером на 1–2 пункта больше, чем в основном тексте.
\makeatletter
\renewcommand\section{%
  \@startsection{section}{1}%
    {\fivecharsapprox}%
    {-1em plus-1ex minus-.2ex}%
    {1em plus.2ex}%
    {\raggedright\hyphenpenalty=10000\normalfont\large\bfseries}}
\makeatother

% Зачем: Задает стиль заголовков подразделов
% Почему: Пункт 4.3.1 Заголовки подразделов печатают с абзацного отступа строчными буквами (кроме первой прописной)
% полужирным шрифтом с размером шрифта основного текста.
\makeatletter
\renewcommand\subsection{
  \@startsection{subsection}{2}%
    {\fivecharsapprox}%
    {-1em plus-1ex minus-.2ex}%
    {1em plus.2ex}%
    {\raggedright\hyphenpenalty=10000\normalfont\normalsize\bfseries}}
\makeatother

% Зачем: Задает стиль заголовков пунктов
% Почему: Пункт 4.3.1 При необходимости заголовок пункта печатают с абзацного отступа полужирным шрифтом с размером
% шрифта основного текста в подбор к тексту.
\makeatletter
\renewcommand\subsubsection{
  \@startsection{subsubsection}{3}
    {\fivecharsapprox}
    {-1em plus-1ex minus-.2ex}
    {1em plus.2ex}
    {\raggedright\hyphenpenalty=10000\normalsize\bfseries}}
\makeatother

% Зачем: нумерация для пунктов
\setcounter{secnumdepth}{3}

% Зачем: точка в конце пунктов
% Почему: Пункт 4.3.1 В конце заголовка пункта ставят точку
% TODO: не сделано. Нужно все менять на titlesec. Точку нужно ставить вручную :)

% Зачем: Работа с колонтитулами
\usepackage{fancyhdr} % пакет для установки колонтитулов
\renewcommand{\footrulewidth}{0pt} % убрать разделительную линию внизу страницы
\renewcommand{\headrulewidth}{0pt} % убрать разделительную линию вверху страницы

% Зачем: Номер страницы проставляют в центре нижней части листа без точки в конце
% Почему: Пункт 4.4.1 На титульном листе номер страницы не ставят, на
% последующих страницах номер проставляют в центре нижней части листа без точки в конце.
\pagestyle{fancy}
\fancyhf{}\fancyfoot{\cfoot\thepage}

% Если опустить настройку оглавления ниже --- будут проблемы с компиляцией :|
% Зачем: настройка оглавления
\usepackage{tocloft}
\setlength{\cftbeforetoctitleskip}{-1em}
\setlength{\cftaftertoctitleskip}{1em}
\setlength{\cftbeforechapskip}{0pt}
\setlength{\cftbeforepartskip}{0pt}

% Печать слова ГЛАВА
\renewcommand{\cftchappresnum}{Глава}
\renewcommand{\cftchapnumwidth}{5em}

% Добавляем точечки после part, chapter, section
\renewcommand{\cftchapfont}{\normalfont}
\renewcommand{\cftchappagefont}{\normalfont}

\renewcommand{\cftpartfont}{\normalfont}
\renewcommand{\cftpartpagefont}{\normalfont}

% Добавляем точечки после part, chapter, section
\renewcommand{\cftpartleader}{\cftdotfill{\cftdotsep}}
\renewcommand{\cftchapleader}{\cftdotfill{\cftdotsep}}
\renewcommand{\cftsecleader}{\cftdotfill{\cftdotsep}}

% Заголовок ОГЛАВЛЕНИЕ
\renewcommand\contentsname{Оглавление}
\renewcommand\cfttoctitlefont{\hfill\normalfont\large\bfseries\MakeUppercase}
\renewcommand{\cftaftertoctitle}{\hfil\hfill}

% Ссылки в содержании и http ссылки
\usepackage[final,hidelinks,unicode]{hyperref}

% Макрос для печати part и chapter прописными буквами
% Да, вложенный if выглядит убого, но это работает :|
\makeatletter
\let\oldcontentsline\contentsline
\def\contentsline#1#2{%
  \expandafter\ifx\csname l@#1\endcsname\l@part
    \expandafter\@firstoftwo
  \else
    \expandafter\@secondoftwo
  \fi
  {%
    \oldcontentsline{#1}{\MakeTextUppercase{#2}}%
  }{%
      \expandafter\ifx\csname l@#1\endcsname\l@chapter
        \expandafter\@firstoftwo
      \else
        \expandafter\@secondoftwo
      \fi
      {%
        \oldcontentsline{#1}{\MakeTextUppercase{#2}}%
      }{%
        \oldcontentsline{#1}{#2}%
      }%
  }%
}
\makeatother

% Зачем: Пакет для вставки картинок
% Примечание: Объяснение, зачем final - http://tex.stackexchange.com/questions/11004/why-does-the-image-not-appear
\usepackage[final]{graphicx}
\DeclareGraphicsExtensions{.pdf,.png,.jpg,.jpeg}

% Зачем: Директория в которой будет происходить поиск картинок
\graphicspath{{figures/}}

% Зачем: Добавление подписей к рисункам. Рисунки нумеруются в пределах главы
% Почему: Пункт 4.4.8 Номер иллюстрации (таблицы) должен состоять из номера
% главы и порядкового номера иллюстрации (таблицы), разделенных точкой.
% Пункт 4.5 Слово «Рисунок», его номер и наименование иллюстрации печатают
% полужирным шрифтом, причем слово «Рисунок», его номер, а также пояснительные
% данные к нему – уменьшенным на 1–2 пункта размером шрифта.
\usepackage[nooneline,figurewithin=chapter,font=small]{caption}

% Зачем: Задание подписей, разделителя и нумерации частей рисунков
% Почему: Пункт 4.4.8 4.4.8 Иллюстрации и таблицы обозначают соответственно
% словами «Рисунок» и «Таблица» и нумеруют последовательно в пределах каждой главы.
\DeclareCaptionLabelFormat{stbfigure}{Рисунок #2}
\DeclareCaptionLabelFormat{stbtable}{Таблица #2}
\DeclareCaptionLabelSeparator{stb}{~--~}
\captionsetup{labelsep=stb}
\captionsetup[figure]{labelformat=stbfigure,justification=centering,labelfont=bf,textfont=bf,belowskip=-10pt}
\captionsetup[table]{labelformat=stbtable,justification=raggedright,format=hang,aboveskip=0pt}
% Зачем: группировка рисунков
\usepackage{subfig}
% Зачем: русские индексы для рисунков
\renewcommand\thesubfigure{\asbuk{subfigure}}

% Зачем: настройки для листингов кода
% необходим пакет python-pygments
\usepackage{minted}
% Зачем: тема с оттенками серого. Удобна для печати
\usemintedstyle{bw}
% Зачем: настройки для выбранных ЯП
\setminted[c++]{fontfamily=tt,fontsize=\small,xleftmargin=1.25cm,breaklines=true,tabsize=4}

% Зачем: преобразовывать текст в верхний регистр командой MakeTextUppercase
\usepackage{textcase}

% Зачем: Переносы в словах с тире.
% Тире в слове заменяем на \hyph: аппаратно\hyph{}программный.
% https://stackoverflow.com/questions/2193307/how-to-get-latex-to-hyphenate-a-word-that-contains-a-dash#
\def\hyph{-\penalty0\hskip0pt\relax}

% Зачем: оформление библиографии через biblatex
\usepackage{csquotes}
\usepackage[
    sorting=none,
    backend=biber,
    defernumbers=true,
    bibstyle=gost-numeric,
    citestyle=gost-numeric,
    language=auto,
    autolang=other
    ] {biblatex}
% файл с библиографией
\addbibresource{bibliography_database.bib}

% Зачем: добавление отдельной категории для авторских публикаций
% Почему: Пункт 4.9 Сведения об использованных в магистерской диссертации
% источниках приводятся в разделе «Библиографический список», включающем
% подразделы «Список использованных источников» и «Список публикаций
% соискателя».
\DeclareBibliographyCategory{AuthorSources}
\addtocategory{AuthorSources}{mol_uch,55_sntk,56_sntk}

% Более близкий к ГОСТ-2003 стиль
\toggletrue{bbx:gostbibliography}

% Зачем: Добавление постфикса "--А" к авторским публикациям
% https://tex.stackexchange.com/questions/540028/biblatex-add-postfix-to-label
% Почему: Пункт 3.9 В списке использованных источников сведения об источниках
% нумеруют арабскими цифрами, а в списке публикаций автора – арабскими цифрами,
% которые через тире дополняются буквой «А.» («авторская») с точкой.
%+++++++++++++++++++++++++++++++++++++++++++++++++++++++++++++++++++++++++++++++
\DeclareFieldFormat{labelprefix}{--#1}

\defbibenvironment{bibliography}
  {\list
     {\printtext[labelnumberwidth]{%
        \printfield{labelnumber}%
        \printfield{labelprefix}}}
     {\setlength{\labelwidth}{\labelnumberwidth}%
      \setlength{\leftmargin}{\labelwidth}%
      \setlength{\labelsep}{\biblabelsep}%
      \addtolength{\leftmargin}{\labelsep}%
      \setlength{\itemsep}{\bibitemsep}%
      \setlength{\parsep}{\bibparsep}}%
      \renewcommand*{\makelabel}[1]{\hss##1}}
  {\endlist}
  {\item}

\makeatletter
\renewbibmacro*{cite:comp:end}{%
  \usebibmacro{cite:dump}%
  \ifnumgreater{\value{cbx@tempcntb}}{-1}
    {\multicitedelim}
    {}%
  \printtext[bibhyperref]{%
    \printfield{labelnumber}%
    \printfield{labelprefix}}}

\renewbibmacro*{cite:comp:inset}{%
  \usebibmacro{cite:dump}%
  \ifnumgreater{\value{cbx@tempcntb}}{-1}
    {\multicitedelim}
    {}%
  \printtext[bibhyperref]{%
    \printfield{labelnumber}%
    \printfield{labelprefix}%
    \printfield{entrysetcount}}%
  \setcounter{cbx@tempcntb}{-1}}

\renewbibmacro*{cite:dump}{%
  \ifnumgreater{\value{cbx@tempcnta}}{0}
    {\ifnumgreater{\value{cbx@tempcnta}}{1}
       {\bibrangedash}
       {\multicitedelim}%
     \bibhyperref[\cbx@lastkey]{%
       \printtext[labelnumber]{\cbx@lastnumber}%
       \ifdef\cbx@lastprefix
         {\printtext[labelprefix]{\cbx@lastprefix}}
         {}}}
    {}%
  \setcounter{cbx@tempcnta}{0}%
  \global\undef\cbx@lastprefix}
\makeatother
%+++++++++++++++++++++++++++++++++++++++++++++++++++++++++++++++++++++++++++++++

% Запретить использование двойных пробелов в конце предложения.
% Почему: это было актуально только в эпоху печатных машинок.
\frenchspacing

% TODO: За этой линией скрываются макросы, которые возможно пригодятся в
% будущем. Комментарии и код из предыдущих проектов
%----------------------------------------------------------------------------------------


% Зачем: чтобы работала \No в новых латехах
\DeclareRobustCommand{\No}{\ifmmode{\nfss@text{\textnumero}}\else\textnumero\fi}

% Зачем: поворот ячеек таблиц на 90 градусов
\usepackage{rotating}
\DeclareRobustCommand{\povernut}[1]{\begin{sideways}{#1}\end{sideways}}

% Зачем: когда в формулах много кириллических символов команда \text{} занимает много места
\DeclareRobustCommand{\x}[1]{\text{#1}}


% Зачем: Окружения для оформления формул
% Почему: Пункт 2.4.7 требований по оформлению пояснительной записки и специфические требования различных кафедр
% Пример использования смотри в course_content.tex, строка 5
\usepackage{calc}
\newlength{\lengthWordWhere}
\settowidth{\lengthWordWhere}{где}
\newenvironment{explanationx}
    {%
    %%% Следующие строки определяют специфические требования разных редакций стандартов. Раскоменнтируйте нужную строку
    %% стандартный абзац, СТП-01 2010
    %\begin{itemize}[leftmargin=0cm, itemindent=\parindent + \lengthWordWhere + \labelsep, labelsep=\labelsep]
    %% без отступа, СТП-01 2013
    \begin{itemize}[leftmargin=0cm, itemindent=\lengthWordWhere + \labelsep , labelsep=\labelsep]%
    \renewcommand\labelitemi{}%
    }
    {%
    %\\[\parsep]
    \end{itemize}
    }

% Зачем: Удобная вёрстка многострочных формул, масштабирующийся текст в формулах, формулы в рамках и др
\usepackage{amsmath}

% Зачем: Поддержка ажурного и готического шрифтов
\usepackage{amsfonts}

% Зачем: amsfonts + несколько сотен дополнительных математических символов
\usepackage{amssymb}

% Зачем: Окружения «теорема», «лемма»
\usepackage{amsthm}

% Зачем: Производить арифметические операции во время компиляции TeX файла
\usepackage{calc}

% Зачем: Производить арифметические операции во время компиляции TeX файла
\usepackage{fp}

% Зачем: Пакет для работы с перечислениями
\usepackage{enumitem}
\makeatletter
 \AddEnumerateCounter{\asbuk}{\@asbuk}{щ)}
\makeatother


% Зачем: Устанавливает символ начала простого перечисления
% Почему: Пункт 2.3.5 Требований по оформлению пояснительной записки.
\setlist{nolistsep}


% Зачем: Устанавливает символ начала именованного перечисления
% Почему: Пункт 2.3.8 Требований по оформлению пояснительной записки.
\renewcommand{\labelenumi}{\asbuk{enumi})}
\renewcommand{\labelenumii}{\arabic{enumii})}

% Зачем: Устанавливает отступ от границы документа до символа списка, чтобы этот отступ равнялся отступу параграфа
% Почему: Пункт 2.3.5 Требований по оформлению пояснительной записки.

\setlist[itemize,0]{itemindent=\parindent+ 2.2ex,leftmargin=0ex,label=--}
\setlist[enumerate,1]{itemindent=\parindent+ 2.7ex,leftmargin=0ex}
\setlist[enumerate,2]{itemindent=\parindent+ \parindent-2.7ex}

% Зачем: Дополнительные возможности в форматировании таблиц
\usepackage{makecell}
\usepackage{multirow}
\usepackage{array}


% Зачем: "Умная" запятая в математических формулах. В дробных числах не добавляет пробел
% Почему: В требованиях не нашел, но в русском языке для дробных чисел используется {,} а не {.}
\usepackage{icomma}

% Зачем: макрос для печати римских чисел
\makeatletter
\newcommand{\rmnum}[1]{\romannumeral#1}
\newcommand{\Rmnum}[1]{\expandafter\@slowromancap\romannumeral#1@}
\makeatother


% Зачем: Управление выводом чисел.
\usepackage{sistyle}
\SIdecimalsign{,}

% Зачем: inline-коментирование содержимого.
\newcommand{\ignore}[2]{\hspace{0in}#2}


% Зачем: Возможность коментировать большие участки документа
\usepackage{verbatim}


\usepackage{xcolor}

\usepackage[normalem]{ulem}

% Моноширинный шрифт выглядит визуально больше, чем пропорциональный шрифт, если их размеры одинаковы. Искусственно уменьшаем размер ссылок.
\renewcommand{\UrlFont}{\normalfont\normalsize}

% Магия для подсчета разнообразных объектов в документе
\usepackage{lastpage}
\usepackage{totcount}
\regtotcounter{section}

\newcounter{totfigures}
\newcounter{tottables}
\newcounter{totreferences}
\newcounter{totequation}

\providecommand\totfig{}
\providecommand\tottab{}
\providecommand\totref{}
\providecommand\toteq{}

\makeatletter
\AtEndDocument{%
  \addtocounter{totfigures}{\value{figure}}%
  \addtocounter{tottables}{\value{table}}%
  \addtocounter{totequation}{\value{equation}}
  \immediate\write\@mainaux{%
    \string\gdef\string\totfig{\number\value{totfigures}}%
    \string\gdef\string\tottab{\number\value{tottables}}%
    \string\gdef\string\totref{\number\value{totreferences}}%
    \string\gdef\string\toteq{\number\value{totequation}}%
  }%
}
\makeatother

\pretocmd{\section}{\addtocounter{totfigures}{\value{figure}}\setcounter{figure}{0}}{}{}
\pretocmd{\section}{\addtocounter{tottables}{\value{table}}\setcounter{table}{0}}{}{}
\pretocmd{\section}{\addtocounter{totequation}{\value{equation}}\setcounter{equation}{0}}{}{}
\pretocmd{\bibitem}{\addtocounter{totreferences}{1}}{}{}



% Для оформления таблиц не влязящих на 1 страницу
\usepackage{longtable}

% Для включения pdf документов в результирующий файл
\usepackage{pdfpages}

% Для использования знака градуса и других знаков
% http://ctan.org/pkg/gensymb
\usepackage{gensymb}

% Нумерованный список с арабскими цифрами на всех уровнях нумерации
\newlist{legal}{enumerate}{10}
\setlist[legal]{font=\bfseries}

\newlist{enum}{enumerate}{10}
\setlist[enum]{label*=\arabic*.,font=\normalfont}

\setlist[enum,1]{itemindent=\parindent+ 2.7ex,leftmargin=0ex}
\setlist[enum,2]{itemindent=\parindent+ \parindent-2.7ex}

% НЕ ДВИГАТЬ. Может поломаться
% Зачем: Удаляет точки после нумерации section и прочих
% Почему: Пункт 4.4.6 Заголовки разделов, подразделов, пунктов приводят после их номеров через пробел.
\AtBeginDocument{%
   \def\postpart{\@aftersepkern}%
   \def\postchapter{\@aftersepkern}%
   \def\postsection{\@aftersepkern}%
   \def\postsubsection{\@aftersepkern}%
   \def\postsubsubsection{\@aftersepkern}%
   \def\postparagraph{\@aftersepkern}%
   \def\postsubparagraph{\@aftersepkern}%
}
