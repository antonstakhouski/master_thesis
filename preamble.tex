% !TeX spellcheck = russian-aot-ieyo
% Зачем: Определяет класс документа (То, как будет выглядеть документ)
% Примечание: параметр draft помечает строки, вышедшие за границы страницы, прямоугольником, в фильной версии его нужно удалить.
\documentclass[a4paper,14pt,russian,oneside,final]{extreport}

% Для мультиязыковой поддержки
\usepackage{polyglossia}

% Установка язков
\setdefaultlanguage{russian}
\setmainlanguage{russian}
\setotherlanguage{english}

% Лигатуры нужны для правильного отображения тире, кавычек и прочего
\setmainfont[Ligatures=TeX]{Times New Roman}
\setmonofont[Ligatures=TeX]{Courier New}

% Необходимо для киррилических шрифтов
\newfontfamily{\cyrillicfont}[Ligatures=TeX]{Times New Roman}
\newfontfamily{\cyrillicfonttt}[Ligatures=TeX]{Courier New}

%% Зачем: Чтобы можно было использовать русские буквы в формулах, но в случае использования предупреждать об этом.
%\usepackage[warn]{mathtext}

% Зачем: Учет особенностей различных языков.
\usepackage{babel}

% Зачем: Длинна, пимерно соответвующая 5 символам
% Почему: Требования содержат странное требование про отсупы в 5 символов (для немоноширинного шрифта :| )
\newlength{\fivecharsapprox}
\setlength{\fivecharsapprox}{6ex}


% Зачем: Добавляет отступы для абзацев.
% Почему: Пункт 2.1.3 Требований по оформлению пояснительной записки.
\usepackage{indentfirst}
\setlength{\parindent}{\fivecharsapprox} % Примерно соответсвует 5 символам.


% Зачем: Настраивает отступы от границ страницы.
% Почему: Пункт 4.2. Положения о магистерской диссертации
\usepackage[left=3cm,top=2.0cm,right=1.0cm,bottom=2.0cm]{geometry}

% Зачем: межстрочный интервал в 18pt
% Почему: Пункт 4.2. Положения о магистерской диссертации
\usepackage{leading}
\leading{18pt}

% Зачем: Отключает использование изменяемых межсловных пробелов.
% Почему: Так не принято делать в текстах на русском языке.
\frenchspacing

% Зачем: Сброс счетчика сносок для каждой страницы
% Примечание: в "Требованиях по оформлению пояснительной записки" не указано, как нужно делать, но в других БГУИРовских докуметах рекомендуется нумерация отдельная для каждой страницы
\usepackage{perpage}
\MakePerPage{footnote}


% Зачем: Добавляет скобку 1) к номеру сноски
% Почему: Пункты 2.9.2 и 2.9.1 Требований по оформлению пояснительной записки.
\makeatletter
\def\@makefnmark{\hbox{\@textsuperscript{\normalfont\@thefnmark)}}}
\makeatother


% Зачем: Расположение сносок внизу страницы
% Почему: Пункт 2.9.2 Требований по оформлению пояснительной записки.
\usepackage[bottom]{footmisc}

% Зачем: Настраивает отступ между таблицей с содержанимем и словом СОДЕРЖАНИЕ
% Почему: Пункт 2.2.7 Требований по оформлению пояснительной записки.
% TODO: make chapter and part uppercase
\usepackage{tocloft}
\setlength{\cftbeforetoctitleskip}{-1em}
\setlength{\cftaftertoctitleskip}{1em}
\setlength{\cftbeforechapskip}{0pt}
\setlength{\cftbeforepartskip}{0pt}

\renewcommand{\cftchappresnum}{Глава }
\renewcommand{\cftchapnumwidth}{4em}

\renewcommand{\cftchapfont}{\normalfont}
\renewcommand{\cftchappagefont}{\normalfont}

\renewcommand{\cftpartfont}{\normalfont}
\renewcommand{\cftpartpagefont}{\normalfont}

\renewcommand{\cftpartleader}{\cftdotfill{\cftdotsep}}
\renewcommand{\cftchapleader}{\cftdotfill{\cftdotsep}}
\renewcommand{\cftsecleader}{\cftdotfill{\cftdotsep}}

\renewcommand\contentsname{Оглавление}
\renewcommand\cfttoctitlefont{\hfill\normalfont\large\bfseries\MakeUppercase}
\renewcommand{\cftaftertoctitle}{\hfil\hfill}

\setcounter{secnumdepth}{3}

% Зачем: Работа с колонтитулами
\usepackage{fancyhdr} % пакет для установки колонтитулов

\pagestyle{fancy}
\fancyhf{}\fancyfoot[R]{\cfoot\thepage}

\renewcommand{\footrulewidth}{0pt} % убрать разделительную линию внизу страницы
\renewcommand{\headrulewidth}{0pt} % убрать разделительную линию вверху страницы

% Зачем: Номер страницы проставляют в центре нижней части листа без точки в конце
% Почему: Пункт 4.4.1 Положения о магистерской диссертации
% TODO: у начала главы нумерация пропадает
\fancypagestyle{plain}{
    \fancyhf{}\fancyfoot[R]{\cfoot\thepage}
}

% Зачем: Установить для large размер в 15пт
% Почему: Пункт 4.3.1 Положения о магистерской диссертации
\makeatletter
\renewcommand\large{\@setfontsize\large{15pt}{18}} % м.б. 18пт baselineskip некорректен
\makeatother

% Зачем: для оформления введения и заключения, они должны быть выровнены по центру.
% Почему: Пункт 4.3.1 Положения о магистерской диссертации
\makeatletter
\renewcommand\part{%
  \clearpage\@startsection{part}{-1}%
    {\fivecharsapprox}%
    {-1em \@plus-1ex \@minus-.2ex}%
    {1em \@plus.2ex}%
    {\centering\hyphenpenalty=10000\normalfont\large\bfseries\MakeUppercase}%
    }
\makeatother

% Зачем: Задает стиль заголовков глав жирным шрифтом, прописными буквами, без точки в конце
% Почему: Пункт 4.3.1 Положения о магистерской диссертации
\usepackage{titlesec}
\titleformat{\chapter}[display]
{\normalfont\large\centering\bfseries}
{\MakeTextUppercase\chaptertitlename\ \thechapter}
{.5em}
{\MakeTextUppercase}

\titlespacing*{\chapter}
  {0pt}{0pt}{1em}

% Зачем: Задает стиль заголовков разделов
% Почему: Пункт 4.3.1 Положения о магистерской диссертации
\makeatletter
\renewcommand\section{%
  \@startsection{section}{1}%
    {\fivecharsapprox}%
    {-1em \@plus-1ex \@minus-.2ex}%
    {1em \@plus.2ex}%
    {\raggedright\hyphenpenalty=10000\normalfont\large\bfseries}}
\makeatother

% Зачем: Задает стиль заголовков подразделов
% Почему: Пункт 4.3.1 Положения о магистерской диссертации
\makeatletter
\renewcommand\subsection{
  \@startsection{subsection}{2}%
    {\fivecharsapprox}%
    {-1em \@plus-1ex \@minus-.2ex}%
    {1em \@plus.2ex}%
    {\raggedright\hyphenpenalty=10000\normalfont\normalsize\bfseries}}
\makeatother

% Зачем: Задает стиль библиографии
% Почему: Пункт 2.8.6 Требований по оформлению пояснительной записки.
\bibliographystyle{styles/ugost2003}


% Зачем: Пакет для вставки картинок
% Примечание: Объяснение, зачем final - http://tex.stackexchange.com/questions/11004/why-does-the-image-not-appear
\usepackage[final]{graphicx}
\DeclareGraphicsExtensions{.pdf,.png,.jpg,.eps}


% Зачем: Директория в которой будет происходить поиск картинок
\graphicspath{{figures/}}


% Зачем: Добавление подписей к рисункам
\usepackage[nooneline]{caption}

% Зачем: чтобы работала \No в новых латехах
\DeclareRobustCommand{\No}{\ifmmode{\nfss@text{\textnumero}}\else\textnumero\fi}

% Зачем: поворот ячеек таблиц на 90 градусов
\usepackage{rotating}
\DeclareRobustCommand{\povernut}[1]{\begin{sideways}{#1}\end{sideways}}


% Зачем: когда в формулах много кириллических символов команда \text{} занимает много места
\DeclareRobustCommand{\x}[1]{\text{#1}}


% Зачем: Задание подписей, разделителя и нумерации частей рисунков
% Почему: Пункт 2.5.5 Требований по оформлению пояснительной записки.
\DeclareCaptionLabelFormat{stbfigure}{Рисунок #2}
\DeclareCaptionLabelFormat{stbtable}{Таблица #2}
\DeclareCaptionLabelSeparator{stb}{~--~}
\captionsetup{labelsep=stb}
\captionsetup[figure]{labelformat=stbfigure,justification=centering}
\captionsetup[table]{labelformat=stbtable,justification=raggedright,format=hang,aboveskip=0pt}
\usepackage{subfig}
\renewcommand\thesubfigure{\asbuk{subfigure}}


% Зачем: Окружения для оформления формул
% Почему: Пункт 2.4.7 требований по оформлению пояснительной записки и специфические требования различных кафедр
% Пример использования смотри в course_content.tex, строка 5
\usepackage{calc}
\newlength{\lengthWordWhere}
\settowidth{\lengthWordWhere}{где}
\newenvironment{explanationx}
    {%
    %%% Следующие строки определяют специфические требования разных редакций стандартов. Раскоменнтируйте нужную строку
    %% стандартный абзац, СТП-01 2010
    %\begin{itemize}[leftmargin=0cm, itemindent=\parindent + \lengthWordWhere + \labelsep, labelsep=\labelsep]
    %% без отступа, СТП-01 2013
    \begin{itemize}[leftmargin=0cm, itemindent=\lengthWordWhere + \labelsep , labelsep=\labelsep]%
    \renewcommand\labelitemi{}%
    }
    {%
    %\\[\parsep]
    \end{itemize}
    }

% Старое окружение для "где". Сохранено для совместимости
\usepackage{tabularx}

\newenvironment{explanation}
    {
    %%% Следующие строки определяют специфические требования разных редакций стандартов. Раскоменнтируйте нужные 2 строки
    %% стандартный абзац, СТП-01 2010
    %\par
    %\tabularx{\textwidth-\fivecharsapprox}{@{}ll@{ --- } X }
    %% без отступа, СТП-01 2013
    \noindent
    \tabularx{\textwidth}{@{}ll@{ --- } X }
    }
    {
    \\[\parsep]
    \endtabularx
    }


% Зачем: Удобная вёрстка многострочных формул, масштабирующийся текст в формулах, формулы в рамках и др
\usepackage{amsmath}


% Зачем: Поддержка ажурного и готического шрифтов
\usepackage{amsfonts}


% Зачем: amsfonts + несколько сотен дополнительных математических символов
\usepackage{amssymb}


% Зачем: Окружения «теорема», «лемма»
\usepackage{amsthm}


% Зачем: Производить арифметические операции во время компиляции TeX файла
\usepackage{calc}

% Зачем: Производить арифметические операции во время компиляции TeX файла
\usepackage{fp}

% Зачем: Пакет для работы с перечислениями
\usepackage{enumitem}
\makeatletter
 \AddEnumerateCounter{\asbuk}{\@asbuk}{щ)}
\makeatother


% Зачем: Устанавливает символ начала простого перечисления
% Почему: Пункт 2.3.5 Требований по оформлению пояснительной записки.
\setlist{nolistsep}


% Зачем: Устанавливает символ начала именованного перечисления
% Почему: Пункт 2.3.8 Требований по оформлению пояснительной записки.
\renewcommand{\labelenumi}{\asbuk{enumi})}
\renewcommand{\labelenumii}{\arabic{enumii})}

% Зачем: Устанавливает отступ от границы документа до символа списка, чтобы этот отступ равнялся отступу параграфа
% Почему: Пункт 2.3.5 Требований по оформлению пояснительной записки.

\setlist[itemize,0]{itemindent=\parindent+ 2.2ex,leftmargin=0ex,label=--}
\setlist[enumerate,1]{itemindent=\parindent+ 2.7ex,leftmargin=0ex}
\setlist[enumerate,2]{itemindent=\parindent+ \parindent-2.7ex}

% Зачем: Включение номера раздела в номер формулы. Нумерация формул внутри раздела.
\AtBeginDocument{\numberwithin{equation}{section}}

% Зачем: Включение номера раздела в номер таблицы. Нумерация таблиц внутри раздела.
\AtBeginDocument{\numberwithin{table}{section}}

% Зачем: Включение номера раздела в номер рисунка. Нумерация рисунков внутри раздела.
\AtBeginDocument{\numberwithin{figure}{section}}


% Зачем: Дополнительные возможности в форматировании таблиц
\usepackage{makecell}
\usepackage{multirow}
\usepackage{array}


% Зачем: "Умная" запятая в математических формулах. В дробных числах не добавляет пробел
% Почему: В требованиях не нашел, но в русском языке для дробных чисел используется {,} а не {.}
\usepackage{icomma}

% Зачем: макрос для печати римских чисел
\makeatletter
\newcommand{\rmnum}[1]{\romannumeral#1}
\newcommand{\Rmnum}[1]{\expandafter\@slowromancap\romannumeral#1@}
\makeatother


% Зачем: Управление выводом чисел.
\usepackage{sistyle}
\SIdecimalsign{,}

% Зачем: inline-коментирование содержимого.
\newcommand{\ignore}[2]{\hspace{0in}#2}


% Зачем: Возможность коментировать большие участки документа
\usepackage{verbatim}


\usepackage{xcolor}


% Зачем: Оформление листингов кода
% Примечание: final нужен для переопределения режима draft, в котором листинги не выводятся в документ.
\usepackage[final]{listings}

\usepackage[normalem]{ulem}

\usepackage[final,hidelinks,unicode]{hyperref}
% Моноширинный шрифт выглядит визуально больше, чем пропорциональный шрифт, если их размеры одинаковы. Искусственно уменьшаем размер ссылок.
\renewcommand{\UrlFont}{\normalfont\normalsize}

\usepackage[square,numbers,sort&compress]{natbib}
\setlength{\bibsep}{0em}

% Магия для подсчета разнообразных объектов в документе
\usepackage{lastpage}
\usepackage{totcount}
\regtotcounter{section}

\newcounter{totfigures}
\newcounter{tottables}
\newcounter{totreferences}
\newcounter{totequation}

\providecommand\totfig{}
\providecommand\tottab{}
\providecommand\totref{}
\providecommand\toteq{}

\makeatletter
\AtEndDocument{%
  \addtocounter{totfigures}{\value{figure}}%
  \addtocounter{tottables}{\value{table}}%
  \addtocounter{totequation}{\value{equation}}
  \immediate\write\@mainaux{%
    \string\gdef\string\totfig{\number\value{totfigures}}%
    \string\gdef\string\tottab{\number\value{tottables}}%
    \string\gdef\string\totref{\number\value{totreferences}}%
    \string\gdef\string\toteq{\number\value{totequation}}%
  }%
}
\makeatother

\pretocmd{\section}{\addtocounter{totfigures}{\value{figure}}\setcounter{figure}{0}}{}{}
\pretocmd{\section}{\addtocounter{tottables}{\value{table}}\setcounter{table}{0}}{}{}
\pretocmd{\section}{\addtocounter{totequation}{\value{equation}}\setcounter{equation}{0}}{}{}
\pretocmd{\bibitem}{\addtocounter{totreferences}{1}}{}{}



% Для оформления таблиц не влязящих на 1 страницу
\usepackage{longtable}

% Для включения pdf документов в результирующий файл
\usepackage{pdfpages}

% Для использования знака градуса и других знаков
% http://ctan.org/pkg/gensymb
\usepackage{gensymb}

% Зачем: преобразовывать текст в верхний регистр командой MakeTextUppercase
\usepackage{textcase}

% Зачем: Переносы в словах с тире.
% Тире в словае заменяем на \hyph: аппаратно\hyphпрограммный.
% https://stackoverflow.com/questions/2193307/how-to-get-latex-to-hyphenate-a-word-that-contains-a-dash#
\def\hyph{-\penalty0\hskip0pt\relax}

% Добавляем абзацный отступ для библиографии
% https://github.com/mstyura/bsuir-diploma-latex/issues/19
\setlength\bibindent{-1.0900cm}

\makeatletter
\renewcommand\NAT@bibsetnum[1]{\settowidth\labelwidth{\@biblabel{#1}}%
   \setlength{\leftmargin}{\bibindent}\addtolength{\leftmargin}{\dimexpr\labelwidth+\labelsep\relax}%
   \setlength{\itemindent}{-\bibindent+\fivecharsapprox-0.240cm}%
   \setlength{\listparindent}{\itemindent}
\setlength{\itemsep}{\bibsep}\setlength{\parsep}{\z@}%
   \ifNAT@openbib
     \addtolength{\leftmargin}{\bibindent}%
     \setlength{\itemindent}{-\bibindent}%
     \setlength{\listparindent}{\itemindent}%
     \setlength{\parsep}{10pt}%
   \fi
}

% Нумерованный список с арабскими цифрами на всех уровнях нумерации
\newlist{legal}{enumerate}{10}
\setlist[legal]{font=\bfseries}

\newlist{enum}{enumerate}{10}
\setlist[enum]{label*=\arabic*.,font=\normalfont}

\setlist[enum,1]{itemindent=\parindent+ 2.7ex,leftmargin=0ex}
\setlist[enum,2]{itemindent=\parindent+ \parindent-2.7ex}

% Зачем: Удаляет точки после нумерации section и прочих
% Почему: Пункт 4.3.1 Положения о магистерской диссертации
\AtBeginDocument{%
   \def\postpart{\@aftersepkern}%
   \def\postchapter{\@aftersepkern}%
   \def\postsection{\@aftersepkern}%
   \def\postsubsection{\@aftersepkern}%
   \def\postsubsubsection{\@aftersepkern}%
   \def\postparagraph{\@aftersepkern}%
   \def\postsubparagraph{\@aftersepkern}%
}

\usepackage{minted}

\setminted[c++]{fontfamily=tt,fontsize=\small,xleftmargin=1.25cm,breaklines=true,tabsize=4}
\setminted[JSON]{fontfamily=tt,fontsize=\small,xleftmargin=1.25cm,breaklines=true,tabsize=2}

\usemintedstyle{bw}

\setlength{\belowcaptionskip}{-10pt}
