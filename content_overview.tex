\part*{Содержание работы}
\addcontentsline{toc}{part}{Содержание работы}

Во \textbf{введении} дается обоснование актуальности работы, приводится краткий
обзор проблематики задачи и современного состояния исследований по симуляции
огня, описывается область применения результатов исследований.

В \textbf{общей характеристике работы} сформулированы ее цель и задачи, даны
сведения об объекте и предмете исследования, приведены апробации и публикации
результатов.

В \textbf{первой главе} произведен анализ предыдущих работ по данному
направлению. В главе приведена краткая историческая справка по развитию онлайн и
оффлайн симуляции, произведен обзор наиболее важных работ по теме диссертации. В
конце главы приводится сравнительный анализ различных алгоритмов симуляции огня.

Во \textbf{второй главе} приводится описание теоретической базы исследования.
Глава начинается с краткого введения в компьютерною графику, в котором
вводятся необходимые определения, далее следует обзор библиотеки
OpenGL, используемой в практической части диссертации. Во втором разделе главы
дается краткое описание основных физических процессов, происходящих в процессе
горения. В последнем разделе главы приводится описание структуры симуляции,
дается обзор особенностей стадий моделирования, анимации и визуализации,
приводится описание особенностей из взаимосвязи.

В \textbf{третьей главе} дается подробное описание используемых в диссертации
алгоритмов. В главе приводятся описание различных проблем, возникших в ходе
создания симулятора, и обзор их решений. В главе приводится множество
иллюстраций, описывающих влияния различных алгоритмов и подходов на систему.
Также в главе приведены результаты экспериментов по оценке влияния количества
частиц в системе на частоту кадров.

В \textbf{заключении} приводится анализ результатов исследования, приводится
описание преимуществ и недостатков работы, описаны способы улучшения системы,
дается описание проблем, которые будут раскрыты в дальнейших исследованиях.
