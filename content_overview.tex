\part*{Содержание работы}
\addcontentsline{toc}{part}{Содержание работы}

Общий объем магистерской диссертации составляет 59 страниц, включая 38
иллюстраций, 4 таблицы, библиографический список из 45 наименований, 1
приложение.

Во \textbf{введении} дается обоснование актуальности работы,
описываются\break{}прикладные задачи, в которых может быть использована
разработанная система, приводится краткий перечень требований к разрабатываемой
системе. Также в разделе приводится краткий обзор проблематики задачи и
современного состояния исследований по симуляции огня, дается перечень наиболее
актуальных проблем в данной области.

В \textbf{общей характеристике работы} сформулированы цель и задачи
исследования, даны сведения об объекте и предмете исследования, приведены
апробации и публикации результатов.

В \textbf{первой главе} произведен анализ предыдущих работ по данному
направлению. В первом разделе приведена краткая историческая справка по развитию
онлайн и оффлайн симуляции. В разделе описана эволюция методов симуляции огня в
кинематографе от первой работы по симуляции огня, использованной в 1982 году в
фильме ''Звездный путь 2: Гнев Хана'' до симуляции в фильме ''Хоббит: Битва пяти
воинств'' 2014 года. Также в данном разделе приводится широкий обзор методов,
применяемых в приложениях реального времени, в частности в видеоиграх. В разделе
приводится критический анализ различных групп методов симуляции огня, приводятся
примеры успешного переноса идей из области оффлайн симуляции в область онлайн
симуляции. Во втором разделе выполнена систематизация рассмотренных методов
онлайн симуляции, приводится классификация методов симуляции огня и
сравнительный анализ полученных классов.

Во \textbf{второй главе} приводится описание теоретической базы исследования.
В первом разделе производится анализ инструментов для рендеринга трехмерной
сцены, приводится обоснования выбора библиотеки OpenGL в качестве инструмента
для рендеринга симуляции. В разделе приводится краткий обзор работы графического
конвейера и вводятся необходимые определения. Во втором разделе
приводятся определения понятий ''огонь'' и ''пламя'', приводится описание
процесса и компонентов горения. В разделе приводится
классификация видов огня и обосновывается выбор класса огня для симуляции. Также
в разделе описываются различные физические атрибуты и характеристики огня,
которые оказывают значительное влияние на поведение и визуальное восприятие
огня. В третьем разделе приводится структура симуляции и дается описание
различных методов моделирования, анимации и визуализации, которые будут
использованы для симуляции огня в данной работе.

В \textbf{третьей главе} приводится подробное описание разработанного
симулятора. В разделе описываются особенности моделирования огня с помощью
системы частиц, представлены алгоритмы для преодоления ограничений данного
метода, описаны особенности реализации данного метода в разработанном
симуляторе. Во втором разделе представлен используемый в работе алгоритм
анимации частиц. В разделе представлены основные уравнения и алгоритмы,
используемые для анимации пламени в рамках диссертации. В третьем разделе
описывается использование текстурных сплэтов для повышения визуального качества
сцены и оптимизации количества частиц. В последнем разделе приведены
экспериментальные данные работы системы и приводится сравнение с результатами,
полученными с помощью более сложной модели.

В \textbf{заключении} приводится краткий обзор результатов, полученных на каждом
из этапов исследования, приводится обоснование выбранных методов и инструментов,
дается критический анализ разработанной системы, и приводится описание проблем,
которые будут более подробно раскрыты в дальнейших исследованиях.
