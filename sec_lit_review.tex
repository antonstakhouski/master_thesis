\section{Обзор литературы}

Моделирование бесформенных объектов, таких как дым, огонь, туман, дымка является
предметом постоянных исследований в области компьютерной графики, поскольку
данные объекты не имеют четко очерченных границ и являются мобильными по своей
природе. Высокая коммерческая ценность данных эффектов в кинематографе и
видеоиграх является двигателем постоянных исследований в данной области.
Наибольший вызов представляет моделирование данных объектов в реальном времени,
где необходимо получить максимально реалистичную симуляцию за время обновления
экранного кадра (60Hz+)~\cite{lec17}.

\subsection{Эволюция алгоритмов симуляции огня}
\subsubsection{Истоки. Система частиц}
\begin{figure}[htb]
	\centering
	\includegraphics[scale=0.55]{st2_init}
	\caption{Первоначальный взрыв}
\end{figure}
Пионером компьютерной симуляции огня является Reewes~W.~T. В 1983 году в своей
работе он ввел понятие система частиц в качестве примитива для моделирования,
анимации и рендеринга~\cite{reewes1983}. В фильме ''Звездный путь 2: Гнев Хана''
он смоделировал так называемую ''расширяющуюся стену огня'', созданную с помощью
двухуровневой системы частиц. Система частиц верхнего уровня находилась в центре
взрыва генезис-бомбы, она генерировала частицы, которые в свою очередь являлись
системами частиц. Эти системы частиц использовались для моделирования взрывов,
при которых каждая такая система частиц вела себя как небольшой вулкан,
извергающийся в сторону распространения взрывной волны и затухающий под
воздействием силы гравитации. Поскольку частицы имеют дискретную природу, для
достижения хороших результатов потребовалось колоссальное количество частиц. Но,
поскольку моделирование в реальном времени не требовалось, это не оказалось
проблемой.
\begin{figure}[htb]
	\centering
	\includegraphics[scale=0.55]{st2wall}
	\caption{Стена огня вот-вот поглотит камеру}
\end{figure}

\subsubsection{Оффлайн симуляция}
В данное время наибольшего успеха исследователи добились в нечувствительном к
времени симуляции кинематографе. Несмотря на то, что данная работа нацелена на
компьютерную графику реального времени, необходимо упомянуть несколько работ в
области оффлайн симуляции, поскольку понимание основных идей, заложенных в них,
позволил перенести некоторые из них в область графики реального времени.

В публикации 2002 года Nguen и его коллеги представили метод моделирования огня,
полностью основанный на физико-математическом подходе~\cite{nguen2002}. В
симуляции использовались несжатые уравнения Навье-Строкса для горячих газов, это
позволило также смоделировать эффект расширения, вызванный испарением, и эффект
текучести поднимающихся дыма и сажи. Как видно на изображении, данная симуляция
отличается реалистичным позиционированием и движением газообразных субстанций.
Однако данный подход сложно реализовать в рендеринге реального времени,
поскольку необходимо находить решение большого количества комплексных уравнений
за время кадра.
\begin{figure}[htb]
	\centering
	\includegraphics[scale=0.5]{nguen1}
	\caption{Два горящих полена находятся на земле и являются источником
 топлива. Бревно, лежащее поперек, еще не загорелось, поэтому пламя его обтекает}
\end{figure}

В 2008 году Horwath H. и Geiger W. представили инновационную комбинацию
симуляции с помощью крупной решетки частиц и тонко настроенных
визуально-ориентированных улучшений симуляции, рассчитываемых на
GPU~\cite{Stock:2008:SWF:1400385.1400457}. Полученные
изображения имеют поразительную детализацию и могут быть легко интегрированы в
кинематографические фотоснимки.

Данная техника улучшения симуляции использует особенности и ограничения
зрительного восприятия, а также особенности концентрации внимания зрителя.
Множество независимых GPU используются для быстрого увеличения качества
изображения, что позволяет достичь очень высокого разрешения.
\begin{figure}[htb]
	\centering
	\includegraphics[scale=0.5]{gpu2008}
	\caption{Три различных кадра симуляции огня. Быстро движущийся огненный
	шар с искрами. Извивающийся костер. Плотная стена дыма и огня.}
\end{figure}

\subsubsection{Онлайн симуляция}
\begin{figure}%
    \centering
    \subfloat[\small{Огонь создан с помощью пререндеренного ядра
	битмапа, которое окружают светящиеся анимированные в реальном
времени частицы}]{{\includegraphics[width=7.5cm]{doom_splat} }}%
    \qquad
    \subfloat[\small{Объемный факел, созданный из непрозрачных полигонов}]
    {{\includegraphics[width=7.5cm]{doom_torch} }}%
    \caption{Скриншоты из Quake (1996)~\cite{capstone}}%
    \label{fig:example}%
\end{figure}
Трехмерный огонь, моделируемый в реальном времени, находит свое применение в
интерактивных приложениях. Среди интерактивных приложений можно выделить
компьютерные игры, в которых необходимость показывать взрывы появилась
практически с самого момента их появления. Компьютерные игры являются основными
потребителями графических компьютерных анимаций огня. Однако это стало возможным
лишь пару десятилетий назад. С тех пор скорость аппаратного обеспечения для
рендеринга время росла экспоненциально, открывая возможности для все более и
более детализированных эффектов. К сожалению, поскольку игры зачастую являются
проприетарными по своей природе, литературных источников по алгоритмам,
используемых в играх крайне мало.
\begin{figure}[htb]
	\centering
	\includegraphics[scale=0.4]{farcry2}
	\caption{Far Cry 2 (2008)~\cite{farcry2}. На момент своего выхода в игре
	была наиболее реалистичная симуляция степных пожаров}
\end{figure}
Компания NVIDIA представила в 2014 году систему NVIDIA FlameWorks. Данная
система позволяет добавлять реалистичный огонь, дым и эффекты взрывов в игры.
Данная система совмещает передовую симуляцию жидкостей на основе решетки вместе
с эффективной системой объемного рендеринга, все оптимизировано для работы в
реальном времени. Все вычисления выполняются на GPU с помощью DirectX
11~\cite{Green:2014:NFR:2633956.2658828}.
\begin{figure}[htb]
	\centering
	\includegraphics[scale=0.9]{green2014}
	\caption{Демонстрация работы NVIDIA FlameWorks}
\end{figure}
\subsection{Обзор и классификация алгоритмов симуляции огня}
Различные методы применяемые при симуляции огня можно условно разделить на
следующие группы:
\begin{itemize}
	\item текстурный маппинг;
	\item система частиц;
	\item физико-математические методы;
	\item клеточные автоматы;
	\item томографическая реконструкция и др.
\end{itemize}

В 2011 году ZhaoHui W., Zhong Z. и Wei W. представили статью~\cite{survey}, в
которой проанализировали
современные алгоритмы симуляции реалистичного огня. Ими был проведен анализ
наиболее популярных методов по следующим критериям:
\begin{itemize}
	\item применимость в реальном времени;
	\item степень реалистичности;
	\item пространственно-временная сложность;
	\item конфигурируемость;
	\item интерактивность.
\end{itemize}
Результаты данного исследования можно увидеть в таблице ниже.

\begin{table}[htb]
    \caption{Сравнение производительности различных методов симуляции огня}
    \includegraphics[scale=0.65]{simulation_methods}
\end{table}

Для использования в магистерской диссертации был выбран физико-математический
подход, поскольку он позволяет получить качественную реалистичную симуляцию с
реалистичным взаимодействием с окружающим миром. Недостатки данного метода в
скорости симуляции возможно сгладить с помощью использования современных GPU,
оптимизации алгоритмов симуляции, уменьшения объемов симуляции (моделирование
небольших источников огня).
